\section*{Conclusion}
\addcontentsline{toc}{section}{Conclusion}

Ce projet nous a permis d'appréhender les notions importantes que sont les intégrations et dérivés numériques. En effet il est capital de faire la distinction entre une instabilité physique et une instabilité numérique qui peut être causé par un mauvais choix de dérivé par exemple. Une partie importante du travail a également nécessité une compréhension théorique du système à simuler. Une autre part du travail a été l'optimisation du code afin que la simulation calcul les équations physiques aussi rapidement que possible.

Une des pistes d'amélioration de cette simulation de disque d’accrétion à une dimension autour d'un trou noir serait d'inclure la singularité à notre système. En effet, les effets relativistes sont pour le moment complètement masqués par une approche képlerienne. En revanche, la validité d'une telle approximation est remise en cause pour des faibles rayons, proches de l'ISCO où les effets gravitationnels ne peuvent pas être négligés. De telles modifications pourraient être prises en compte par un feuilletage de l'espace-temps et une réécriture des équations dans un tel système. Une autre piste d'amélioration serait l'utilisation d'un modèle plus précis pour le disque d'accrétion que le modèle $\alpha$. Cela nécessiterait probablement une résolution de la turbulence dans le disque et mènerait à un code beaucoup plus complexe. Par ailleurs, il serait aussi possible d'inclure des perturbations magnétiques, comme par exemple un modèle approximé de la MRI (Magneto-Rotationnal Instability). La diversité des phénomènes négligés dans ce modèle impose d'étudier analytiquement le disque et les résultats que nous avons présentés afin d'éliminer les phénomènes ayant un impact plus faible. 
%%% Local Variables:
%%% mode: latex
%%% TeX-master: "rapport.tex"
%%% End: