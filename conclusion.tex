\section*{Conclusion}
\addcontentsline{toc}{section}{Conclusion}


Ce projet nous a permis d'appréhender les notions importantes que sont les intégrations et dérivés numériques. En effet il est capital de faire la distinction entre une instabilité physique et une instabilité numérique qui peut être causé par un mauvais choix de dérivé par exemple. Une partie importante du travail a également nécessité une compréhension théorique du sytème a simulé (voir \ref{sec::heatmap}) Enfin les enjeux plus technique consernant l'optimisation du temps de calcul qui peut s'avérer indispensable pour une éventuelle progression de ce projet.\\


Un des enjeux pour développer cette simulation de disque d'accretion à une dimension autour d'un trou noir serait d'include la singularité à notre système. En effet elle n'intervient que comme condition au bord intéreur pour résoudre les équations sur $T$, $\Sigma$ et $v$. Cela s'avererait cependant d'une complexité bien supérieure puisque nous ne pourions plus considérer le disque en rotation képleienne aux abords du trou noir. Par cet exemple consernant la grandeur physique $\Omega$, nous voyons rapidemment la necessité de prendre en compte des effets relativistes pour comprendre comment la matière est accrétée à l'intérieur du trou noir. Bien que les trous noirs soient les objets les plus simple de l'Univers d'un point de vue des paramètres qui le caractérises, la physique de son environnement n'en reste pas moins complexe. 