\section*{Conclusion}
\addcontentsline{toc}{section}{Conclusion}


\FIXME{ce que nous avons appris > importance des schémas numériques ..} \\


Un des enjeux pour développer cette simulation de disque d'accretion à une dimension autour d'un trou noir serait d'include la singularité à notre système. En effet elle n'intervient que comme condition au bord intéreur pour résoudre les équations sur $T$, $\Sigma$ et $v$. Cela s'avererait cependant d'une complexité bien supérieure puisque nous ne pourions plus considérer le disque en rotation képleienne aux abords du trou noir. Par cet exemple consernant la grandeur physique $v$, nous voyons rapidemment la necessité de prendre en compte des effets relativistes pour comprendre comment la matière est accrétée à l'intérieur du trou noir. Bien que les trous noirs soient les objets les plus simple de l'Univers d'un point de vue des paramètres qui le caractérises, la physique de son environnement n'en reste pas moins complexe. 