\subsection{Intégration numérique de S et T}

\subsubsection{Intégration implicite de $S$}

\label{ssec:integration_S_imp}

L'intégration implicite revient à calculer $S^t$ en fonction de $S^{t+1}$, ce
qui donne un système linéaire qu'il est ensuite possible d'inverser. Il est
important de noter que cette méthode est plus coûteuse à chaque pas de temps
qu'une méthode explicite, car elle nécessite l'inversion d'une matrice. Il faut
donc vérifier que le gain de temps de calcul lié à une intégration sur un plus
grand pas de temps est plus important que le surcoût lié à l'inversion du
système.\FIXME{le faire …}. Dans la simulation, on vérifie systématiquement que
le temps de relaxation de $T$ est plus faible que le temps visqueux.

\paragraph{Linéarisation des équations}

On souhaite intégrer $S^\star$, la densité surfacique adimensionnée. L'équation
d'évolution est :

\begin{equation}
  \frac{\partial S^\star}{\partial t^\star} = \frac{1}{x^2}\frac{\partial^2}{\partial x^2}\left(\nu^\star S^\star\right)
\end{equation}

Pour alléger les notations, nous allons ommettre les étoiles dans les
développements prochains. L'équation s'écrit alors en passant aux variables
discrètes, au temps $t$ et à la case $1<n\leq n_\textrm{max}$

\begin{equation}
  \label{eq:S_discret_n}
  \frac{S^{t+1}_n - S^t_n}{\Delta t} = \frac{1}{x_n^2}\frac{\nu^t_{n+1}S^{t+1}_{n+1} - 2 \nu^t_nS^{t+1}_n + \nu^t_{n-1}S^{t+1}_{n-1}}{\Delta x^2}
\end{equation}

À l'intérieur du disque, la densité de matière est supposée nulle
($\nu^t_{0}S^{t+1}_{0} = 0$). Physiquement, il est en effet attendu qu'aucune
orbite keplerienne circulaire ne soit stable en dessous de l'ISCO
(\emph{\emph{I}nnermost \emph{S}table \emph{C}ircular \emph{O}rbit}), située en
$r = 3M$. En approximation, on suppose donc que la densité de matière y est
nulle.

\begin{equation}
  \label{eq:S_discret_1}
  \frac{S^{t+1}_1 - S^t_1}{\Delta t} = \frac{1}{x_1^2}\frac{\nu^t_{2}S^{t+1}_{2} - 2 \nu^t_1S^{t+1}_1}{\Delta x^2}
\end{equation}

Pour la case $n = n_\textrm{max} = N$, on a d'après \eqref{eq:nuS_n_is_null}:

\begin{equation}
  \frac{\nu^{t}_{N+1}S^{t+1}_{N+1} - \nu^{t}_NS^{t+1}_N}{\Delta x} = \dot{M}^\star_N
\end{equation}

Il faut noter que $\dot{M}^\star_N$ vaut initialement $1$. Pour simuler
l'arrivée de matière par l'extérieur du disque, cette quantité est incrémentée
au fur et à mesure de la simulation. Donc l'équation \eqref{eq:S_discret_n}
s'écrit en $N$:

\begin{equation}
  \label{eq:S_discret_N}
  \frac{S^{t+1}_N - S^t_N}{\Delta t} = \frac{1}{x_N^2}\frac{\Delta x - \nu^{t}_NS^{t+1}_N + \nu^{t}_{N-1}S^{t+1}_{N-1}}{\Delta x^2}
\end{equation}

\paragraph{Écriture matricielle}

On réécrit \eqref{eq:S_discret_n}, \eqref{eq:S_discret_1} et
\eqref{eq:S_discret_N} pour exprimer $S^t$ en fonction de $S^{t+1}$:

\begin{equation}
  \left\lbrace\begin{array}{r l c l c l }
    S^{t}_1 = &
               & &S_1^{t+1}\left(1 + 2\frac{\Delta t}{\Delta x^2}\frac{\nu_1^t}{x_1^2}\right)
               &+& S_{2}^{t+1}\left(-\frac{\Delta t}{\Delta x^2}\frac{\nu_{2}^t}{x_1^2}\right)\\
    S^{t}_n = &S_{n-1}^{t+1}\left(-\frac{\Delta t}{\Delta x^2}\frac{\nu_{n-1}^t}{x_n^2}\right)
               &+& S_n^{t+1}\left(1 + 2\frac{\Delta t}{\Delta x^2}\frac{\nu_n^t}{x_n^2}\right)
               &+& S_{n+1}^{t+1}\left(-\frac{\Delta t}{\Delta x^2}\frac{\nu_{n+1}^t}{x_n^2}\right)\\
    S^{t}_N + \frac{\Delta t}{\Delta x x_N^2} =
               &S^{t+1}_{N-1} \left(-\frac{\Delta t}{\Delta x^2}\frac{\nu_{N-1}^t}{x_N^2}\right)
               &+& S_N^{t+1} \left(1 + \frac{\Delta t}{\Delta x^2}\frac{\nu^t_N}{x_N^2}\right)
               & &
  \end{array}\right.
\end{equation}

Pour simplifier, on notera $A_k = 1 + 2 \frac{\Delta t}{\Delta x^2}\frac{\nu_k^t}{x_k^2}$ et $\Delta = \frac{\Delta t}{\Delta x^2}$
Ce système d'équation s'écrit aussi sous forme matricielle :
\begin{equation}
  \left(S^t\middle) + 
  \middle(\begin{matrix}
    0 \\
    \\
    \\
    \vdots \\
    \\
    \\
    \\
    0 \\
    \frac{\Delta t}{\Delta x}\frac{1}{x_N^2}
  \end{matrix}\middle)
  =
  \begin{pmatrix}
A_1                            & -\Delta\frac{\nu_{2}^t}{x_1^2} &  & & & & 0\\
-\Delta \frac{\nu_{1}^t}{x_2^2} & A_2                           & -\Delta\frac{\nu_{3}^t}{x_2^2} & & & &\\
    &        & \ddots                          &  & & & &\\
    &        & -\Delta \frac{\nu_{k-1}^t}{x_k^2} & A_k    & -\Delta \frac{\nu_{k+1}^t}{x_k^2} & &\\
    &        &                                 & & \ddots                          & & \\
    & & & & -\Delta \frac{\nu_{N-2}^t}{x_{N-1}^2} & A_{N-1} & -\Delta \frac{\nu_{N}^t}{x_{N-1}^2}\\
    0 & & & & & -\Delta \frac{\nu_{N-1}^t}{x_N^2} & 1 + \Delta \frac{\nu_N^t}{x_N^2}
  \end{pmatrix} \middle(S^{t+1}\right)
\end{equation}

Qui s'écrit aussi :

\begin{equation}
  AS^{t+1} = S^t + X 
\end{equation}

$A$ est une matrice tri-diagonale qu'il est facile de résoudre en utilisant Lapack (\href{http://www.netlib.org/lapack/explore-html/d4/d62/group__double_g_tsolve.html#ga2bf93f2ddefa5e671866eb2191dc19d4}{routine DGTSV}).
\FIXME{ajouter cela comme une référence}

\subsubsection{Intégration explicite de $S$}
\label{ssec:integration_S_exp}

Au contraire de l'intégration implicite, l'intégration explicite permet une
résolution directe d'une équation aux dérivées partielles. Il suffit de
connaître les variables à un temps $t$ pour en déduire la nouvelle valeur au
temps $t+1$ :

\begin{equation}
  S^{t+1} = S^t + \Delta t f(t)
\end{equation}
La solution au temps suivant est directement donnée.

\subsubsection{Intégration explicite de $T$}
\label{ssec:integration_T}

Il est possible d'améliorer le schéma d'intégration explicite pour la
température en donnant une meilleure estimation de $f(t)$.

\begin{equation}
  \frac{\partial T}{\partial t} = \frac{Q^+ - Q^- + Q_\textrm{adv}}{C_V} = f(T, S)
\end{equation}

On peut écrire un développement limité de $f(T) = f_0 + \frac{\partial
f}{\partial T}\Delta T + \frac{\partial f}{\partial S}\Delta S$ Comme on a un
temps thermique très petit devant le temps visqueux $\tau_\nu \gg \tau_T$
\FIXME{ est-ce vraiment vrai ?} , les variations de $f$ par rapport à $S$ sont
très faibles devant celles par rapport à $T$. On obtient alors $f(T) = f_0 +
f'_T \Delta T$. En linéarisant, on obtient finalement: \begin{equation} T^{t+1}
= T^t + \frac{f_0}{f'_T}\left[e^{f'_T\Delta t} - 1 \right] \end{equation} On
peut facilement obtenir numériquement une approximation de $f'_T$ en calculant
la quantité $\frac{f(T+\delta T) - f(T)}{\delta T}$, avec $\delta T = \lambda
T,\ \lambda \ll 1$.

\FIXME{ ajouter plot qui montre qu'on a bien df/dS << df/dT}
\FIXME{ajouter plot qui montre que la courbe est suffisamment lisse, c'est-à-dire que f'T peut bien être approximé numériquement}


