\section{Résolution numérique}

\subsection{Principe de la résolution}

\subsection{Courbe en S}

\subsection{Considérations sur les dérivées}

\begin{equation}
    \frac{\partial u}{\partial t} + v \frac{\partial u}{\partial r} = 0
\end{equation}

\begin{align}
    \frac{\partial u_j}{\partial r} &\approx \frac{u_{j+1} - u_j}{\Delta{r}}
    \frac{\partial u_j}{\partial r} &\approx \frac{u_j - u_{j-1}}{\Delta{r}}
    \frac{\partial u_j}{\partial r} &\approx \frac{u_{j+1} - u_{j-1}}{\Delta{r}}
\end{align}

\begin{equation}
    \frac{\partial \tilde{u}}{\partial t} = - v \frac{\exp(ik\Delta{r}) - \exp(-ik\Delta{r})}{2 \Delta{r}} \tilde{u}
\end{equation}

\begin{align}
    \exp(ik\Delta{r}) &= 1 + ik\Delta{r} + \frac{i^2 k^2 \Delta{r}^2}{2}
    \exp(-ik\Delta{r}) &= 1 - ik\Delta{r} + \frac{i^2 k^2 \Delta{r}^2}{2}
\end{align}

\begin{equation}
    \frac{\partial \tilde{u}}{\partial t} = - v \frac{2ik\Delta{r}}{2\Delta{r}} \tilde{u} = - i k v \tilde{u}
\end{equation}

\begin{equation}
    \tilde{u} = \tilde{u(0)} \exp(-i k v t)
\end{equation}

En fait, nous avons tout simplement :

\begin{equation}
    \frac{\exp(ik\Delta{r}) - \exp(-ik\Delta{r})}{2 \Delta{r}} = \frac{2 i \sin(k\Delta{r})}{2 \Delta{r}} = i k \sinc(k\Delta{r})
\end{equation}

Soit :
\begin{equation}
    \frac{\partial \tilde{u}}{\partial t} = - i k v \sinc(k\Delta{r}) \tilde{u}
\end{equation}

\begin{equation}
    \tilde{u} = \tilde{u(0)} \exp(-i k v \sinc(k \Delta{r}) t)
\end{equation}
