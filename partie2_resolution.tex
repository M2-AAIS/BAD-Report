\section{Résolution numérique}

Dans cette partie, nous allons expliquer comment le problème se résout
numériquement et aborder quelques considérations en rapport avec cette
résolution.

\subsection{Principe de la résolution}
La résolution peut être découpée en plusieurs partie, plus ou moins
indépendantes.

Une première considération numérique est la discrétisation spatiale du problème
: le disque sera représenté par un ensemble de position discrète selon l’axe
radial. Ces positions s’étendent de $x_{min} = \sqrt{r_{min}/r_s}$ à $x_{max} =
\sqrt{r_{max}/r_s}$, et sont séparées par un $dx$ constant, ce qui aura pour
effet une fois revenu dans le monde dimensionné d’avoir un $dr$ parabolique,
avec donc plus de points vers le bord intérieur du disque et moins vers le bord
extérieur. Comme l'instabilité commence par se développer au niveau du bord intérieur, il est favorable d'y avoir un bon échantillonnage.

\subsubsection{Courbe en S}

L’idée ici est de trouver les zones de stationnarité du système, c’est-à-dire
celles pour lesquelles la température évolue peu. Cela revient à résoudre
l’équation $Q^+ = Q^-$. En pratique, cette courbe a une forme « en S », avec
une partie arrondie correspondant au cas optiquement épais et une partie droite
correspondant au cas optiquement mince. Il y a deux points critiques : un au
point extrême de la partie arrondie, appelé premier point critique, et un au
point de jonction des deux parties, appelé second point critique.

Il faudra donc déterminer la courbe de stationnarité en chaque point du disque,
puis sortir les coordonnées des points critiques pour chacun d’entre eux ainsi
que déterminer une position initiale pour le système sur ces courbes.

Cette partie a été réalisée essentiellement par Maximilien et Michelle.

\subsubsection{Domaines de la simulation}

Le disque à un moment $t$ peut être dans l'une des trois situations suivantes.

\paragraph{Régime stationnaire} ce régime correspond au disque dans une situation telle que tout le flux de matière entrant traverse le disque pour finir par tomber dans le trou noir. Il est caractérisé par une température et une densité surfacique constantes au cours du temps. Le taux d'accrétion est alors constant et égal au taux d'accrétion à son bord extérieur. Numériquement, il se traduit par une évolution relative de $T$ et $S$ très faible entre deux pas de temps. Lorsque ce régime est atteint, on augmentera le flux de matière entrant au bord externe et ce jusqu'à ce qu'il n'y ait plus de régime stationnaire.

\paragraph{Régime en évolution stable} ce régime correspond au disque évoluant sur un temps visqueux pour la densité surfacique et un temps thermique pour la température. Dans ce régime, on pourra négliger le terme advectif du chauffage. Ce régime se situe dans la partie inférieure de la courbe en S et est caractérisé par un quasi-équilibre thermique. On aura donc $Q^+ = Q^-$. Dans un diagramme $\Sigma-T$, ce régime correspond au suivi de la partie inférieure droite de la courbe en S.
\paragraph{Régime en évolution instable} ce régime est celui de l'instabilité. Il est caractérisé par une évolution de la densité surfacique et de la température sur un même temps thermique. Dans ce régime, au moins une partie du disque se situe à une température ou une densité de surface plus élevée que les valeurs critiques. L'équilibre thermique n'y est alors pas atteint. C'est là que se situe le cœur du problème. Le disque évacue de l'énergie par rayonnement et perds beaucoup de matière par chute dans le trou noir. Chaque point proche du point critique va effectuer des cycles autour du point critique. Ces points voient d'abord leur densité augmentée jusqu'à dépasser la densité critique. Au dela, la température croît rapidement jusqu'à ce que l'advection ne soit plus négligeable et limite la température tout en diminuant la densité localement. Le point se dirige alors vers la gauche, au dessus du point critique jusqu'à retraverser la courbe en S, où le refroidissement est plus important que le chauffage. Le point retombe alors dans la partie inférieure de la courbe en S. Tant qu'il existe des points instables, ils vont se stabiliser en déstabilisant leurs voisins.

\subsubsection{Évolution du disque}

La simulation commence dans le régime en évolution stable et avec des paramètres tels qu'un régime stationnaire existe. Le disque va alors rapidement se stabiliser sur la courbe en S avant de converger lentement vers l'état stationnaire. Une fois l'état stationnaire atteint, le flux de matière est augmenté. On recommence alors à suivre la courbe en S jusqu'à atteindre un nouveau régime stationnaire.

Lorsque le flux de matière entrant est trop élevé, il n'existe plus de régime stationnaire mais seulement un régime en évolution instable. Les points proches des points critiques commencent alors à effectuer des cycles d'amplitude croissante jusqu'à déclencher l'instabilité. Une fois l'instabilité terminée, la densité et la température sont de nouveau assez faible pour retomber dans un régime d'évolution stable.

\subsubsection{Intégration numérique de S et T}

L'évolution du disque d'accrétion est dominé par les deux équations sur $S$ \eqref{eq:difS} et sur $T$ \eqref{eq:difT}.

Les régimes stationnaires et en évolution stable seront traités avec un schéma explicite pour la température et un schéma implicite pour la densité de surface. On a alors un temps visqueux $\tau_\nu$ \FIXME{référence à une courbe, une section?} (voir section \ref{sec::pas_de_temps}) très grand devant le temps thermique $\tau_T$. On peut donc intégrer $S$ et $T$ sur deux échelles de temps différentes et effectuer une démarche de ``step and relax'', où on intègre d'abord $S$ sur un grand temps (visqueux) avant de laisser se relaxer $T$ sur des petits temps (thermiques). Il faut à cet endroit faire attention à ce que le temps de relaxation de $T$ soit toujours plus petit que le temps visqueux. Cela revient à vérifier que le nombre d'itérations nécessaires à la stabilisation de $T$ est plus petit que $\nicefrac{\tau_\nu}{\tau_T}$. Si cette condition est violée, la simulation numérique ne traduit plus de réalité physique. 

Une méthode implicite pour $S$ est alors préconisée. En effet, celle-ci assure de converger vers la solution et est stable quelque soit le pas de temps (dans la limite des approximations données précédemment). Les détails de la méthode sont donnés dans la partie \ref{ssec:integration_S_imp}. Pour intégrer la température, il n'est pas possible d'écrire un schéma implicite. Il faudra donc utiliser un schéma explicite. En revanche, il sera possible de procéder à des simplifications pour donner une convergence plus rapide de $T$. Les détails de ces approximations sont donnés dans la partie \ref{ssec:integration_T}.

L'autre régime est instable. Les évolutions de $S$ et de $T$ se font sur des temps similaires. Dans ce régime, il n'est plus légitime de traiter séparement ces deux variables et il faut donc résoudre conjointement l'une et l'autre. Le schéma implicite n'y est pas utilisé, car il n'assure que la convergence vers une solution physique, mais pas via des états successifs physiquement acceptables. Comme la valeur de $S$ influe sur celle de $T$, cela impliquerait que l'intégration de $T$ n'est pas non plus exacte. Dans le cas du comportement chaotique de l'instabilité, il est nécessaire de limiter au maximum les erreurs numériques car celles-ci peuvent mener à un résultat final différent du résultat physiquement attendu. Le choix d'un schéma explicite est alors fait pour $T$ et $S$, les deux évoluant alors sur le plus petit temps charactéristique du système : le pas de temps thermique. Le schéma explicite est donné dans la partie \ref{ssec:integration_S_exp} pour $S$ et dans la partie \ref{ssec:integration_T} pour $T$.


\subsubsection{Considérations sur les dérivées numériques spatiales}

Dans un certain nombre d’équations interviennent plusieurs dérivées spatiales,
du premier ou second ordre. Il existe plusieurs manières de les calculer
numériquement, ayant des conséquences variées. Nous nous intéresserons à
celles-ci dans cette partie afin d’expliquer les choix effectués.

\subsubsection{Conditions aux bords}

Toujours à propos des dérivées spatiales intervenant, l’existence de bords «
numériques » au problème implique d’imposer des conditions aux bords pour ces
dernières. Ces conditions dépendent des dérivées numériques utilisées, dans
cette partie seront présentées uniquement celles intervenant pour les dérivées
effectivement utilisées, les autres seront présentées en ANNEXE CL.
%TODO: Annexe CL

\subsubsection{Calcul des autres variables}

%\subsection{Courbe en S}
%\section{Détermination des courbes en S}

[INTRO BLABLA]
\\   

\subsection{Etablissement de la fonction}

Au premier abord, pour obtenir une seule courbe en S, le rayon r a été fixé. Ainsi la vitesse angulaire \Omega a été préalablement définie. Par la suite nous avons répété la démarche suivante pour 256 differentes valeurs du rayon.
\\
Pour définir la température T en fonction de \Sigma, nous avons d' abord essayé de simplifier l' égalité $Q^+$ = $Q^-$ , qui traduit l' équilibre thermique local, en remplaçant tous les termes par T,\Sigma et \Omega. Celà donne deux expressions (selon l'opacité du milieu) de fonctions implicites où ces variables sont couplées.
La résolution n' étant pas plus simple qu' avant le changement et afin d' éviter toute erreur induite lors du remplacement des différentes variables impliquées, nous avons préferé de calculer indépendemment chaque variable dans un ordre bien choisi. 
\\
En première partie, nous avons calculé la demi-hauteur H en utilisant la résolution d' une équation quadratique ne dépendant que de (T,\Sigma,\Omega) (Voir \cref{Equation}).
\\
Ensuite, la densité volumique \rho, la vitesse du son $c_s$ et la viscosité \nu. ( ref)
\\
De plus $\kappa_{ff}$, $\kappa_{e}$ et $\epsilon_{ff}$ pour pouvoir calculer le flux radiatif $F_z$ et déduire la différence des deux termes de chaleurs (en ne prenant pas en compte le terme d' advection) :
\\
\begin{equation} 
  \label{eq:qplus-qmoins}
   $Q^+$ - $Q^-$ = 0. 
\end{equation}
\\
Nous avons aussi déterminé la profondeur optique effective $\tau_{eff}$. Selon sa valeur, nous pouvons nous placer dans un cas dit optiquement épais ($\tau_{eff}$ > 1) ou bien dans le cas optiquement mince  ($\tau_{eff}$ < 1).
Cependant nous ne l' avons pas utilisé comme critère pour le choix de l' expression du flux pour éviter de tomber sur des solutions stables artificielles qui seraient le fruit d' interpolations aux alentours de $\tau_{efff}$ = 1 où a lieu le basculement entre les deux approximations utilisées. 
\\
Nous avons donc calculé \cref{eq:qplus-qmoins} dans deux cas où on a imposé l' expression du flux radiatif. Ainsi, nous avons un algorithme où l' on peut fixer les valeurs de T et \Omega pour obtenir une fonction ne dépendant que de \Sigma.


\subsection{Intervalle de température}

\subsection{Résolution}
\subsubsection{Méthode de la dichotomie}

Pour déterminer la densité de surface associée à une valeur de température précise, nous avons procédé par dichotomie.
Cette méthode consiste à trouver les coordonnées où s' annulle une fonction continue en partant d' un intervalle de départ (voir \ref{}) et en le divisant successivement pour cerner l' intervalle le plus petit possible où se trouve la solution.  
\\
Soit f(T,\Sigma) = $Q^+$ - $Q^-$ = 0,  une fonction continue et strictement monotone dans un intervalle [$\Sigma_{min}$ ; $\Sigma_{max}$] pour un T donné. Si f(T,[$\Sigma_{min}$ ) et  f(T,[$\Sigma_{max}$ ) sont de signes opposés, alors d' après le théorème de la bijection il existe une solution unique comprise dans cette intervalle. 
\\
Nous avons appliqué cette bissection en divisant à chaque fois l' intervalle en deux jusqu' à obtenir une solution avec une précision de $10^-6$ et en limitant le nombre d' itérations pour des raisons pratiques. Nous avons alors obtenu deux graphes, représentant T en fonction de \Sigma pour un milieu optiquement épais (tracé courbé) et pour un milieu optiquement mince (relation linéaire).
\\
[deux graphes log T-logS]

\subsubsection{Autres méthodes}

\subsection{Points critiques}
Afin de déterminer la propagation de l' instabilité thermique, il faut trouver le point critique où le système devient instable. Au niveau de la courbe en S, ceci a lieu sur la branche inférieure au niveau de la courbure : au-delà de ce point la densité surfacique diminue tandis que la température continue à s' élever.
\\
Pour trouver les coordonnées de ce point critique, nous avons comparé les valeurs de \Sigma (calculé pour un milieu optiquement épais) pour chaque point en augmentant T. Lorsque la variation de la courbe change, c' est-à-dire quand \Sigma cesse d' augmenter et commence à diminuer, nous avons repéré les coordonnées du dernier point pour lequel la densité surfacique croît.
\\
[?discussion sur les autres manières envisagées ?]        
\\
Dans le code, nous avons nommé "deuxième point critique", le point où a lieu le basculement d'épaisseur optique. Sur les graphes, ce point se situe à l' intersection des deux branches et signale le changement de régime où on a utilisé une approximation de diffusion à une autre approximation où l'on doit prendre en compte les pertes par rayonnement de freinage (le refroidissement par bremsstrahlung).
\\
 Pour le repérer, nous avons eu la même approche que pour pour le point d' instabilité trouvé préalablement. Cette fois-ci, nous avons comparé le \Sigma déterminé pour un milieu optiquement épais à celui optiquement mince. Quand le premier devient plus petit que le second, celà signifie que la densité surfacique arrête de décroître et qu' elle commence à augmenter linéairement en fonction de la température. 
\\
[?si on mentionne la façon avec les dérivées : ce point là est un point anguleux!]
\\
[courbe en S avec les points légendés + ?evolution en fonction de r ?]

\subsection{Construction de la courbe}

\subsection{Evolution du $\tau_eff$}
[To be continued...]

\\
[DISCUSSION BLABLA]


\subsection{Intégration numérique de S et T}\subsection{Intégration implicite de $S$}
\label{ssec:integration_S_imp}
L'intégration implicite revient à calculer $S^t$ en fonction de $S^{t+1}$, ce qui donne un système linéaire qu'il est ensuite possible d'inverser. Il est important de noter que cette méthode est plus coûteuse à chaque pas de temps qu'une méthode explicite, car elle nécessite l'inversion d'une matrice. Il faut donc vérifier que le gain de temps de calcul lié à une intégration sur un plus grand pas de temps est plus important que le surcoût lié à l'inversion du système.\FIXME{le faire …}. Dans la simulation, on vérifie systématiquement que le temps de relaxation de $T$ est plus faible que le temps visqueux.

\paragraph{Linéarisation des équations}

On souhaite intégrer $S^\star$, la densité surfacique adimensionnée. L'équation d'évolution est :
\begin{equation}
  \frac{\partial S^\star}{\partial t^\star} = \frac{1}{x^2}\frac{\partial^2}{\partial x^2}\left(\nu^\star S^\star\right)
\end{equation}
Pour alléger les notations, nous allons ommettre les étoiles dans les développements prochains. L'équation s'écrit alors en passant aux variables discrètes, au temps $t$ et à la case $1<n\leq n_\textrm{max}$
\begin{equation}
  \label{eq:S_discret_n}
  \frac{S^{t+1}_n - S^t_n}{\Delta t} = \frac{1}{x_n^2}\frac{\nu^t_{n+1}S^{t+1}_{n+1} - 2 \nu^t_nS^{t+1}_n + \nu^t_{n-1}S^{t+1}_{n-1}}{\Delta x^2}
\end{equation}
À l'intérieur du disque, la densité de matière est supposée nulle ($\nu^t_{0}S^{t+1}_{0} = 0$). Physiquement, il est en effet attendu qu'aucune orbite keplerienne circulaire ne soit stable en dessous de l'ISCO (\emph{\emph{I}nnermost \emph{S}table \emph{C}ircular \emph{O}rbit}), située en $r = 3M$. En approximation, on suppose donc que la densité de matière y est nulle.
\begin{equation}
  \label{eq:S_discret_1}
  \frac{S^{t+1}_1 - S^t_1}{\Delta t} = \frac{1}{x_1^2}\frac{\nu^t_{2}S^{t+1}_{2} - 2 \nu^t_1S^{t+1}_1}{\Delta x^2}
\end{equation}
Pour la case $n = n_\textrm{max} = N$, on a d'après \eqref{eq:nuS_n_is_null}:
\begin{equation}
  \frac{\nu^{t}_{N+1}S^{t+1}_{N+1} - \nu^{t}_NS^{t+1}_N}{\Delta x} = \dot{M}^\star_N
\end{equation}
Il faut noter que $\dot{M}^\star_N$ vaut initialement $1$. Pour simuler l'arrivée de matière par l'extérieur du disque, cette quantité est incrémentée au fur et à mesure de la simulation.
Donc l'équation \eqref{eq:S_discret_n} s'écrit en $N$:
\begin{equation}
  \label{eq:S_discret_N}
  \frac{S^{t+1}_N - S^t_N}{\Delta t} = \frac{1}{x_N^2}\frac{\Delta x - \nu^{t}_NS^{t+1}_N + \nu^{t}_{N-1}S^{t+1}_{N-1}}{\Delta x^2}
\end{equation}

\paragraph{Écriture matricielle}
On réécrit \eqref{eq:S_discret_n}, \eqref{eq:S_discret_1} et \eqref{eq:S_discret_N} pour exprimer $S^t$ en fonction de $S^{t+1}$:
\begin{equation}
  \left\lbrace\begin{array}{r l c l c l }
    S^{t}_1 = &
               & &S_1^{t+1}\left(1 + 2\frac{\Delta t}{\Delta x^2}\frac{\nu_1^t}{x_1^2}\right)
               &+& S_{2}^{t+1}\left(-\frac{\Delta t}{\Delta x^2}\frac{\nu_{2}^t}{x_1^2}\right)\\
    S^{t}_n = &S_{n-1}^{t+1}\left(-\frac{\Delta t}{\Delta x^2}\frac{\nu_{n-1}^t}{x_n^2}\right)
               &+& S_n^{t+1}\left(1 + 2\frac{\Delta t}{\Delta x^2}\frac{\nu_n^t}{x_n^2}\right)
               &+& S_{n+1}^{t+1}\left(-\frac{\Delta t}{\Delta x^2}\frac{\nu_{n+1}^t}{x_n^2}\right)\\
    S^{t}_N + \frac{\Delta t}{\Delta x x_N^2} =
               &S^{t+1}_{N-1} \left(-\frac{\Delta t}{\Delta x^2}\frac{\nu_{N-1}^t}{x_N^2}\right)
               &+& S_N^{t+1} \left(1 + \frac{\Delta t}{\Delta x^2}\frac{\nu^t_N}{x_N^2}\right)
               & &
  \end{array}\right.
\end{equation}

Pour simplifier, on notera $A_k = 1 + 2 \frac{\Delta t}{\Delta x^2}\frac{\nu_k^t}{x_k^2}$ et $\Delta = \frac{\Delta t}{\Delta x^2}$
Ce système d'équation s'écrit aussi sous forme matricielle :
\begin{equation}
  \left(S^t\middle) + 
  \middle(\begin{matrix}
    0 \\
    \\
    \\
    \vdots \\
    \\
    \\
    \\
    0 \\
    \frac{\Delta t}{\Delta x}\frac{1}{x_N^2}
  \end{matrix}\middle)
  =
  \begin{pmatrix}
A_1                            & -\Delta\frac{\nu_{2}^t}{x_1^2} &  & & & & 0\\
-\Delta \frac{\nu_{1}^t}{x_2^2} & A_2                           & -\Delta\frac{\nu_{3}^t}{x_2^2} & & & &\\
    &        & \ddots                          &  & & & &\\
    &        & -\Delta \frac{\nu_{k-1}^t}{x_k^2} & A_k    & -\Delta \frac{\nu_{k+1}^t}{x_k^2} & &\\
    &        &                                 & & \ddots                          & & \\
    & & & & -\Delta \frac{\nu_{N-2}^t}{x_{N-1}^2} & A_{N-1} & -\Delta \frac{\nu_{N}^t}{x_{N-1}^2}\\
    0 & & & & & -\Delta \frac{\nu_{N-1}^t}{x_N^2} & 1 + \Delta \frac{\nu_N^t}{x_N^2}
  \end{pmatrix} \middle(S^{t+1}\right)
\end{equation}
Qui s'écrit aussi :
\begin{equation}
  AS^{t+1} = S^t + X 
\end{equation}
$A$ est une matrice tri-diagonale qu'il est facile de résoudre en utilisant Lapack (\href{http://www.netlib.org/lapack/explore-html/d4/d62/group__double_g_tsolve.html#ga2bf93f2ddefa5e671866eb2191dc19d4}{routine DGTSV}).
\FIXME{ajouter cela comme une référence}

\subsection{Intégration explicite de $S$}

\label{ssec:integration_S_exp}
Au contraire de l'intégration implicite, l'intégration explicite permet une résolution directe d'une équation aux dérivées partielles. Il suffit de connaître les variables à un temps $t$ pour en déduire la nouvelle valeur au temps $t+1$ :
\begin{equation}
  S^{t+1} = S^t + \Delta t f(t)
\end{equation}
La solution au temps suivant est directement donnée.

\subsection{Intégration explicite de $T$}
\label{ssec:integration_T}
Il est possible d'améliorer le schéma d'intégration explicite pour la température en donnant une meilleure estimation de $f(t)$.
\begin{equation}
  \frac{\partial T}{\partial t} = \frac{Q^+ - Q^- + Q_{adv}}{C_V} = f(T, S)
\end{equation}
On peut écrire un développement limité de $f(T) = f_0 + \frac{\partial f}{\partial T}\Delta T + \frac{\partial f}{\partial S}\Delta S$
Comme on a un temps thermique très petit devant le temps visqueux $\tau_\nu \gg \tau_T$ \FIXME{ est-ce vraiment vrai ?}
, les variations de $f$ par rapport à $S$ sont très faibles devant celles par rapport à $T$. On obtient alors $f(T) = f_0 + f'_T \Delta T$. En linéarisant, on obtient finalement:
\begin{equation}
  T^{t+1} = T^t + \frac{f_0}{f'_T}\left[e^{f'_T\Delta t} - 1 \right] 
\end{equation}
On peut facilement obtenir numériquement une approximation de $f'_T$ en calculant la quantité $\frac{f(T+\delta T) - f(T)}{\delta T}$, avec $\delta T = \lambda T,\ \lambda \ll 1$. 
\FIXME{ ajouter plot qui montre qu'on a bien df/dS << df/dT}
\FIXME{ajouter plot qui montre que la courbe est suffisamment lisse, c'est-à-dire que f'T peut bien être approximé numériquement}

\subsection{Considérations sur les dérivées}
\FIXME{peut être bouger cette partie pour la cohérence ?}

\begin{equation}
    \frac{\partial u}{\partial t} + v \frac{\partial u}{\partial r} = 0
\end{equation}

\begin{align}
    \frac{\partial u_j}{\partial r} &\approx \frac{u_{j+1} - u_j}{\Delta{r}}
    \frac{\partial u_j}{\partial r} &\approx \frac{u_j - u_{j-1}}{\Delta{r}}
    \frac{\partial u_j}{\partial r} &\approx \frac{u_{j+1} - u_{j-1}}{\Delta{r}}
\end{align}

\begin{equation}
    \frac{\partial \tilde{u}}{\partial t} = - v \frac{\exp(ik\Delta{r}) - \exp(-ik\Delta{r})}{2 \Delta{r}} \tilde{u}
\end{equation}

\begin{align}
    \exp(ik\Delta{r}) &= 1 + ik\Delta{r} + \frac{i^2 k^2 \Delta{r}^2}{2}
    \exp(-ik\Delta{r}) &= 1 - ik\Delta{r} + \frac{i^2 k^2 \Delta{r}^2}{2}
\end{align}

\begin{equation}
    \frac{\partial \tilde{u}}{\partial t} = - v \frac{2ik\Delta{r}}{2\Delta{r}} \tilde{u} = - i k v \tilde{u}
\end{equation}

\begin{equation}
    \tilde{u} = \tilde{u(0)} \exp(-i k v t)
\end{equation}

En fait, nous avons tout simplement :

\begin{equation}
    \frac{\exp(ik\Delta{r}) - \exp(-ik\Delta{r})}{2 \Delta{r}} = \frac{2 i \sin(k\Delta{r})}{2 \Delta{r}} = i k \sinc(k\Delta{r})
\end{equation}

Soit :
\begin{equation}
    \frac{\partial \tilde{u}}{\partial t} = - i k v \sinc(k\Delta{r}) \tilde{u}
\end{equation}

\begin{equation}
    \tilde{u} = \tilde{u(0)} \exp(-i k v \sinc(k \Delta{r}) t)
\end{equation}

\subsection{Considérations sur les dérivées numériques}

\begin{equation}
    \frac{\partial u}{\partial t} + v \frac{\partial u}{\partial r} = 0
\end{equation}

\begin{align}
    \frac{\partial u_j}{\partial r} &\approx \frac{u_{j+1} - u_j}{\Delta{r}} \\
    \frac{\partial u_j}{\partial r} &\approx \frac{u_j - u_{j-1}}{\Delta{r}} \\
    \frac{\partial u_j}{\partial r} &\approx \frac{u_{j+1} - u_{j-1}}{2 \Delta{r}}
\end{align}

\begin{equation}
    \frac{\partial \tilde{u}}{\partial t} = - v \frac{\exp(ik\Delta{r}) - \exp(-ik\Delta{r})}{2 \Delta{r}} \tilde{u}
\end{equation}

\begin{align}
    \exp(ik\Delta{r}) &= 1 + ik\Delta{r} + \frac{i^2 k^2 \Delta{r}^2}{2}
    \exp(-ik\Delta{r}) &= 1 - ik\Delta{r} + \frac{i^2 k^2 \Delta{r}^2}{2}
\end{align}

\begin{equation}
    \frac{\partial \tilde{u}}{\partial t} = - v \frac{2ik\Delta{r}}{2\Delta{r}} \tilde{u} = - i k v \tilde{u}
\end{equation}

\begin{equation}
    \tilde{u} = \tilde{u(0)} \exp(-i k v t)
\end{equation}

En fait, nous avons tout simplement :

\begin{equation}
    \frac{\exp(ik\Delta{r}) - \exp(-ik\Delta{r})}{2 \Delta{r}} = \frac{2 i \sin(k\Delta{r})}{2 \Delta{r}} = i k \sinc(k\Delta{r})
\end{equation}

Soit :
\begin{equation}
    \frac{\partial \tilde{u}}{\partial t} = - i k v \sinc(k\Delta{r}) \tilde{u}
\end{equation}

\begin{equation}
    \tilde{u} = \tilde{u(0)} \exp(-i k v \sinc(k \Delta{r}) t)
\end{equation}

\subsection{Conditions aux bords}

\subsection{Calcul des autres variables}

%%% Local Variables:
%%% mode: latex
%%% TeX-master: "rapport"
%%% End: