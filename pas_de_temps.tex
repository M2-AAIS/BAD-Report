\section{Détermination du pas de temps\label{sec::pas_de_temps}}
Nous allons expliciter dans cette section les choix que nous avons fait pour déterminer les différents pas de temps 	(thermique et visqueux) nécessaire aux intégrations en schéma implicite et explicite. \\


Sur la première branche (intégration en schéma implicite), ils sont déterminés au cours de la simulation avant chaque nouvelle intégration en $S$, à partir des formules suivantes : 

\begin{equation}
	\Delta t_{visc}^{1} = \frac{t_{visc}}{K_{visc}} = \frac{1}{K_{visc}} \times \frac{r}{v}
\end{equation}

\begin{equation}
	\Delta t_{th}^{1} = \frac{t_{th}}{K_{th}}= \frac{1}{K_{th}} \times \frac{C_{V} T}{Q^{+} - Q^{-}}
\end{equation} \\

où $K_{visc}$ et $K_{th}$ sont des constantes que l'on à prise respectivement égales à 100 et 1000. \\


Il est nécessaire dans cette simulation de faire intervenir un pas de temps adaptatif. En effet, l'approche du point critique que l'on peut voir en figure \ref{Fig::Stable} est délicate. Le pas de temps visqueux est donc multiplier
à chaque nouvelle intégration sur $S$ par un facteur dépendant de la distance au point critique par rapport à la variable $S$. 


\begin{equation}
	\Delta t_{vis}^{2} = \alpha \Delta t_{visc}^{1}
\end{equation} \\

où $\alpha = 1 - \left( 0.99 \times \e^{- \lvert \frac{S_{crit} - S}{D_{crit}} \rvert} \right)$, avec $D_{crit}$ défini arbitrairement : $D_{}crit = 200$ \\

Le passage en shéma d'intégration explicite est soumis à deux conditions : $T \ge T_{crit}$ ou $\Sigma \ge \Sigma_{crit}$. On bascule dès lors qu'un des rayons à rempli ces conditions. $T$ et $\Sigma$ sont alors intégrés sur le même pas de temps : $0.01 \times \Delta t_{th}^{1}$.
