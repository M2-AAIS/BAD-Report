\subsection{Résolution des équations}

\subsubsection{Résolution de la hauteur $H$ \label{sec::res_H}}
Nous avons vu précédemment que toutes les quantités physiques nécessaires à la simulation sont obtenues à partir de formules simples et immédiates, à l'exception de la vitesse locale d'accrétion $v$ ainsi que la hauteur $H$ du disque.  $H$ fait en effet intervenir la résolution d'un trinôme du second degré dont nous avons détaillé le calcul ci dessous. \\

Partons de la pression P : 

\begin{equation}
	P = P_{gaz} + P_{rad}
\end{equation}


\begin{equation}
	c_{s}^{2} \rho = \frac{R \rho T}{\mu} + \frac{1}{3} a T^{4}
\end{equation}


\begin{equation}
	\Omega^{2} H^{2} \rho = \frac{R T}{\mu} \rho + \frac{1}{3} a T^{4}
\end{equation}


\begin{equation}
	\frac{1}{2} \Omega^{2} \Sigma H = \frac{R T}{\mu} \frac{\Sigma}{2 H} + \frac{1}{3} a T^{4}
	\label{eq:trinôme}
\end{equation}  	   

H est donc solution d'un trinôme du second degré $ a H^{2} + b H + c = 0$. \\
Compte tenu de l'adimensionnement considéré dans cette simulation, nous l'exprimons sous la forme suivante :  

\begin{eqnarray}
  a^{\prime} H^{* 2}+b^{\prime}H^{*}+c^{\prime}=0
  \end{eqnarray}

$\begin{cases}  
      a^{\prime} &= \Omega^{*2} \Omega_{0}^{2} S^{*} S_{0}\\
      b^{\prime} &= - \frac{2 a}{3 r_{s}} T^{*4} T_{0}^{4} x \\
      c^{\prime}&=-\frac{RT_{0}}{\mu r_{s}^{2}} T^{*} S^{*} S_{0}
     \end{cases}$ \\


\begin{equation}
	\Delta^{\prime} = b^{\prime 2} - 4a^{\prime}c^{\prime}
\end{equation} 

Il vient donc : 

\begin{equation}
	\frac{-b -sign(b) \sqrt{\Delta}}{2a}
\end{equation}

\subsubsection{Résolution de la vitesse locale d'accrétion $v$}
\begin{equation}
    v^\star = - \frac{1}{S^\star x} \frac{\partial}{\partial x} \left(\nu^\star S^\star\right)
\end{equation} 

Nous avons privilégié pour cette équation le calcul d'une moyenne des dérivés amonts et avales. La question de savoir lesquelles des dérivées (amonts ou avales) conduisent à des solutions stables n'est pas nécessaire ici puisque en effet il ne s'agit pas d'une équation d'advection mais seulement d'une dérivée spatiale.  


\begin{equation}
    v^\star_{i} = - \frac{1}{S_{i}^\star x_{i}} \frac{( (\nu^{*}S^{*})_{i+1} - (\nu^{*}S^{*})_{i-1} )}{2 \Delta x_{i}}
\end{equation} 

avec comme conditions au bord intérieur : $(\nu^{*}S^{*})_{0} = 0$ et au bord extérieur : $v^\star_{max}   = - \dot{M^{*}}_{max} / (S^{*}x^{*})_{max}$

\subsubsection{Ordre de résolution des équations}

Nous avons donc maintenant toutes les équations  nécessaires à la résolution du système. L'ordre de résolution des équations est le suivant :  \\


        \noindent $(S^{*},T^{*}) \rightarrow (a, b, c, P_{rad}^{*}) \rightarrow H^{*}
        \rightarrow (c_{s}^{*}, \rho^{*}) \\
        c_{s}^{*} \rightarrow \nu^{*} \rightarrow (v^{*}, \dot{M}^{*}) \\
        \rho \rightarrow (P_{gaz}^{*}, \kappa_{ff}, \epsilon) , \\
        P_{gaz}^{*} \rightarrow \beta \\
        \kappa_{ff} \rightarrow \tau_{eff} \rightarrow F_{z} $
        

