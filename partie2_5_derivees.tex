\subsection{Considérations sur les dérivées numériques spatiales}

Il existe un grand nombre de façon de dériver numériquement une variable $u$ en
fonction de l’espace discrétisé selon $r$ (pas spatial $\Delta{r}$). La plus
classique d’entre elles est celle des différences finies au premier ordre, qui
admet trois expressions standards~\eqref{eq:differences_finies}, qui exprime
$u_j = u(r,t)$ en fonctions des valeurs aux positions spatiales adjacentes
$u_{j+1} = u(r+\Delta{r},t)$ et $u_{j-1} = u(r-\Delta{r},t)$.

\begin{subequations}
    \label{eq:differences_finies}
    \begin{align}
        \label{eq:diff_droite}
        \frac{\partial u_j}{\partial r} &\approx \frac{u_{j+1} - u_j    }{  \Delta{r}}\\
        \label{eq:diff_gauche}
        \frac{\partial u_j}{\partial r} &\approx \frac{u_j     - u_{j-1}}{  \Delta{r}}\\
        \label{eq:diff_centre}
        \frac{\partial u_j}{\partial r} &\approx \frac{u_{j+1} - u_{j-1}}{2 \Delta{r}}
    \end{align}
\end{subequations}

Elles sont appelées respectivement \textit{différence à
droite}~\eqref{eq:diff_droite}, \textit{différence à
gauche}~\eqref{eq:diff_gauche} et \textit{différence
centrée}~\eqref{eq:diff_centre}. En pratique, la plus fidèle est celle des
différences centrées, mais elle n’est pas toujours utilisable.

En effet, considérons l’équation d’advection~\eqref{eq:advection_th} sous une
forme assez générale, où $v$ représente la vitesse d’advection (et donc son
signe le sens du gradient). 

\begin{equation}
    \label{eq:advection_th}
    \frac{\partial u}{\partial t} + v \frac{\partial u}{\partial r} = 0
\end{equation}

L’\cref{eq:advection_th} étant linéaire, on peut étudier la stabilité des
solutions numériques selon la méthode de Von Neumann. On considère $u(r,t)$ une
solution du problème et on ajoute une perturbation $\delta{u}$ à celle-ci. On
va alors regarder l’évolution de cette perturbation au cours du temps. La
linéarité nous permet de plus de considérer uniquement un terme de la
transformée de Fourier de $\delta{u}$, qu’on écrit alors $\delta{u} =
\tilde{u}(t) \e^{ikr}$. En réinjectant dans \eqref{eq:advection_th} et en
remplaçant $\frac{\partial u}{\partial r}$ par les expressions données
\cref{eq:differences_finies}, on obtient respectivement :

\begin{subequations}
    \begin{align}
        \frac{\partial \tilde{u}}{\partial t} \e^{ik\Delta{r}} &= - v \frac{\e^{ik(r+\Delta{r})} - \e^{ik r           }}{  \Delta{r}} \tilde{u}\\
        \frac{\partial \tilde{u}}{\partial t} \e^{ik\Delta{r}} &= - v \frac{\e^{ik r           } - \e^{ik(r-\Delta{r})}}{  \Delta{r}} \tilde{u}\\
        \frac{\partial \tilde{u}}{\partial t} \e^{ik\Delta{r}} &= - v \frac{\e^{ik(r+\Delta{r})} - \e^{ik(r-\Delta{r})}}{2 \Delta{r}} \tilde{u}
    \end{align}
\end{subequations}

En simplifiant ces équations par $\e^{ik\Delta{r}}$, cela devient :

\begin{subequations}
    \begin{align}
        \frac{\partial \tilde{u}}{\partial t} &= - v \frac{\e^{ik\Delta{r}} - 1                }{  \Delta{r}} \tilde{u}\\
        \frac{\partial \tilde{u}}{\partial t} &= - v \frac{1                - \e^{-ik\Delta{r}}}{  \Delta{r}} \tilde{u}\\
        \frac{\partial \tilde{u}}{\partial t} &= - v \frac{\e^{ik\Delta{r}} - \e^{-ik\Delta{r}}}{2 \Delta{r}} \tilde{u}
    \end{align}
\end{subequations}

On fait alors un développement limité en $ik\Delta{r}$ :

\begin{align}
    \e^{ ik\Delta{r}} &= 1 + ik\Delta{r} + \frac{i^2 k^2 \Delta{r}^2}{2}\\
    \e^{-ik\Delta{r}} &= 1 - ik\Delta{r} + \frac{i^2 k^2 \Delta{r}^2}{2}
\end{align}

Et on réinjecte :

\begin{subequations}
    \begin{align}
        \frac{\partial \tilde{u}}{\partial t} &= \left(- vik + v\frac{k^2}{2}\right) \tilde{u}\\
        \frac{\partial \tilde{u}}{\partial t} &= \left(- vik - v\frac{k^2}{2}\right) \tilde{u}\\
        \frac{\partial \tilde{u}}{\partial t} &= \left(- vik + 0             \right) \tilde{u}
    \end{align}
\end{subequations}

Ce qui donne en intégrant

\begin{subequations}
    \begin{align}
        \tilde{u} = \tilde{u(0)} \e^{-i k v t} \e^{ v\frac{k^2}{2}t}\\
        \tilde{u} = \tilde{u(0)} \e^{-i k v t} \e^{-v\frac{k^2}{2}t}\\
        \tilde{u} = \tilde{u(0)} \e^{-i k v t}
    \end{align}
\end{subequations}

En fait, nous avons tout simplement pour le cas centré :

\begin{equation}
    \frac{\e^{ik\Delta{r}} - \e^{-ik\Delta{r}}}{2 \Delta{r}} = \frac{2 i \sin(k\Delta{r})}{2 \Delta{r}} = i k \sinc(k\Delta{r})
\end{equation}

Soit :

\begin{equation}
    \frac{\partial \tilde{u}}{\partial t} = - i k v \sinc(k\Delta{r}) \tilde{u}
\end{equation}

Et donc :

\begin{equation}
    \tilde{u} = \tilde{u(0)} \e^{-i k v \sinc(k \Delta{r}) t}
\end{equation}

Finalement :

\begin{subequations}
    \begin{align}
        \tilde{u} = \tilde{u(0)} \e^{-i k v t} \e^{ v\frac{k^2}{2}t}\\
        \tilde{u} = \tilde{u(0)} \e^{-i k v t} \e^{-v\frac{k^2}{2}t}\\
        \tilde{u} = \tilde{u(0)} \e^{-i k v \sinc(k \Delta{r}) t}
    \end{align}
\end{subequations}

Donc dans les deux premiers cas selon le signe de $v$ la perturbation croît ou
s’atténue exponentiellement (et il faudra alors toujours prendre la dérivée
\textit{amont}, celle pour laquelle l’exponentielle est négative pour un $v$
donné, c’est-à-dire en revenant à la définition celle qui résout l’advection
dans le sens du gradient des valeurs hautes aux valeurs basses), et pour la
dérivée centrée, elle reste d’amplitude constante.

Pour résumer cela, on écrira la dérivée amont sous la forme générique suivante :

\begin{equation}
    \frac{\partial u_j}{\partial r} = \frac{\max(v,0) \left[u_{j+1} - u_j \right] + \min(v,0) \left[u_j - u_{j-1}\right]}{\Delta{r}}
\end{equation}


