\section{Equation}
Pour la résolution de H : utilisation du trinôme du second degré.
\begin{eqnarray}
  a^{\prime} H^{* 2}+b^{\prime}H^{*}+c^{\prime}=0
  \end{eqnarray}

  $\begin{cases}  
    a^{\prime} &= \Omega^{*2} \Omega_{0}^{2} S^{*} S_{0}\\
      b^{\prime} &= - \frac{2 a}{3 r_{s}} T^{*4} T_{0}^{4} x \\
        c^{\prime}&=-\frac{RT_{0}}{\mu r_{s}^{2}} T^{*} S^{*} S_{0}
        \end{cases}$\\

        $\Delta^{\prime} = b^{\prime 2} - 4a^{\prime}c^{\prime}$\\

        Cela s'écrit en Fortran : $x_1 = -0.5d0 * (b^{\prime} + sign(sqrt(\Delta^{\prime}),b^{\prime}))/a^{\prime}$\\

        Ordre de résolution : \\


        \noindent $(S^{*},T^{*}) \rightarrow (a, b, c, P_{rad}^{*}) \rightarrow H^{*}
        \rightarrow (c_{s}^{*}, \rho^{*}) \\
        c_{s}^{*} \rightarrow \nu^{*} \rightarrow (v^{*}, \dot{M}^{*}) \\
        \rho \rightarrow (P_{gaz}^{*}, \kappa_{ff}, \epsilon) , \\
        P_{gaz}^{*} \rightarrow \beta \\
        \kappa_{ff} \rightarrow \tau_{eff} \rightarrow F_{z} $
