\subsection{Calcul des autres variables}

Pour chaque temps t donné, nous devons résoudre le système physique en chaque
point du disque. Cela correspond à résoudre un système de 16 équations et 16
inconnus \ref{sec::eq_adim}. Nous avons vu précédemment que toutes ces
quantités physiques nécessaires à  la simulation sont obtenues à  partir de
formules simples et immédiates, à  l'exception de la vitesse locale d'accrétion
$v$ ainsi que la demi-hauteur $H$ du disque.

\subsubsection{Résolution de la demi-hauteur $H$}
\label{ssec::resolution_H}

  $H$ fait en effet intervenir la résolution d'un trinôme du second degré dont nous avons détaillé le calcul ci dessous. \\

Partons de la pression P : 

\begin{equation}
	P = P_\textrm{gaz} + P_\textrm{rad}
\end{equation}


\begin{equation}
	c_{s}^{2} \rho = \frac{R \rho T}{\mu} + \frac{1}{3} a T^{4}
\end{equation}


\begin{equation}
	\Omega^{2} H^{2} \rho = \frac{R T}{\mu} \rho + \frac{1}{3} a T^{4}
\end{equation}


\begin{equation}
	\frac{1}{2} \Omega^{2} \Sigma H = \frac{R T}{\mu} \frac{\Sigma}{2 H} + \frac{1}{3} a T^{4}
\end{equation}

H est donc solution d'un trinôme du second degré $ a H^{2} + b H + c = 0$. \\
Compte tenu de l'adimensionnement considéré dans cette simulation, nous l'exprimons sous la forme suivante :  

\begin{eqnarray}
  a^{\prime} H^{* 2}+b^{\prime}H^{*}+c^{\prime}=0
  \end{eqnarray}

$\begin{cases}  
      a^{\prime} &= \Omega^{*2} \Omega_{0}^{2} S^{*} S_{0}\\
      b^{\prime} &= - \frac{2 a}{3 r_{s}} T^{*4} T_{0}^{4} x \\
      c^{\prime}&=-\frac{RT_{0}}{\mu r_{s}^{2}} T^{*} S^{*} S_{0}
     \end{cases}$ \\


\begin{equation}
	\Delta^{\prime} = b^{\prime 2} - 4a^{\prime}c^{\prime}
\end{equation} 

Il vient donc : 

\begin{equation}
	H^{*} = \frac{-b -sign(b) \sqrt{\Delta}}{2a}
\end{equation}


\subsubsection{Résolution de la vitesse locale d'accrétion $v$}
\begin{equation}
    v^\star = - \frac{1}{S^\star x} \frac{\partial}{\partial x} \left(\nu^\star S^\star\right)
\end{equation} 

Nous avons privilégié pour cette équation le calcul d'une moyenne des dérivées amonts et avales. La question de savoir lesquelles des dérivées (amonts ou avales) conduisent à  des solutions stables n'est pas nécessaire ici puisque en effet il ne s'agit pas d'une équation d'advection mais seulement d'une dérivée spatiale. Ci-dessous l'équation de la dérivée selon cette approximation : 

\begin{equation}
v^\star_{i} = - \frac{1}{S_{i}^\star x_{i}} \frac{( (\nu^{*}S^{*})_{i+1} - (\nu^{*}S^{*})_{i-1} )}{2 \Delta x_{i}}
\end{equation} 

Par contre ce choix de dérivée impose que nous mettions des conditions à  chaque bords du disque. La condition au bord intérieur est  $(\nu^{*}S^{*})_{0} = 0$, tandis que celle au bord extérieur est $v^\star_\textrm{max}   = - \dot{M^{*}}_\textrm{max} / (S^{*}x^{*})_\textrm{max}$.


\subsubsection{Ordre de résolution des équations}

Nous avons donc maintenant toutes les équations  nécessaires à  la résolution du système. L'ordre de résolution des équations est le suivant :  \\

 $\rightarrow$ Une fois le calcul de $H^{*}$ effectué, nous pouvons calculer les valeurs de la vitesse du son $cs^{*}$, de la viscosité $\nu^{*}$ ainsi que de la densité $\rho^{*}$ grâce aux formules décrites précédemment.   

 $\rightarrow$ Le calcul de $v$ nous permet quand à lui de trouver $\dot{M}$. Nous calculons également les valeurs de $\epsilon$,  $\kappa_\textrm{ff}$ et $\tau$, qui servent ensuite dans le calcul de $Fz$.
 
 $\rightarrow$ Enfin nous résolvons les équations des pressions $P_\textrm{gaz}$ et $P_\textrm{rad}$, de $\beta$,  de la capacité $Cv$ ainsi que des termes du transfert thermique $Q^{+}$, $Q^{-}$ et $Q_\textrm{adv}$.

