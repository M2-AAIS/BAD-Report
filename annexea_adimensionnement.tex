\section{Calculs sur l’adimensionnement}
\label{app:adimensionnement}

\subsection{Différentielles}

Les différentielles découlent directement des adimensionnements :
\begin{subequations}
    \begin{align}
        \label{eq:dt_star}
        \partial t &= t_0 \partial t^{\star} \\
        \label{eq:dr_star}
        \partial r &= 2 r_s x \partial x \\
        \label{eq:dT_star}
        \partial T &= T_0 \partial T^{\star}
    \end{align}
\end{subequations}

\subsection{Évolution de la densité de surface}

Nous partons de l’\cref{eq:densite_surface} :
\begin{equation}
    \frac{\partial \Sigma}{\partial t} = \frac{3}{r} \frac{\partial}{\partial r} \left( r^\frac{1}{2} \frac{\partial}{\partial r} \left( \nu \Sigma r^\frac{1}{2} \right) \right)
\end{equation}

Et nous remplaçons $\Sigma$ dans le membre de gauche à l’aide de
l’\cref{eq:definition_S} et $t$ en vertu de l’\cref{eq:dt_star} :
\begin{align}
    \frac{\partial \Sigma}{\partial t} &= \frac{1}{t_0} \frac{\partial}{\partial t^\star} \left( \frac{S}{x} \right) = \frac{1}{x t_0} \frac{\partial S}{\partial t^\star} \\
    \hookrightarrow \frac{\partial S}{\partial t^\star} &= \frac{3 x t_0}{r} \frac{\partial}{\partial r} \left( r^\frac{1}{2} \frac{\partial}{\partial r} \left( \nu \Sigma r^\frac{1}{2} \right) \right)
\end{align}

Cela est dû au fait que $x$ et $t^\star$ sont des variables indépendantes
(puisque $r$ et $t$ le sont).

Puis dans le membre de droite :
\begin{align}
    \Sigma &= \frac{S}{x} = S \left( \frac{r_s}{r} \right)^\frac{1}{2} \Rightarrow \Sigma r^\frac{1}{2} = S r_s^\frac{1}{2} \\
    \hookrightarrow \frac{\partial S}{\partial t^\star} &= \frac{3 x t_0}{r} \frac{\partial}{\partial r} \left( (r_s r)^\frac{1}{2} \frac{\partial}{\partial r} \left( \nu S \right) \right) \\
\end{align}

Nous remplaçons la première dérivée selon $r$ suivant \eqref{eq:dr_star} :
\begin{align}
    \frac{\partial}{\partial r} = \frac{1}{2 r_s x} \frac{\partial}{\partial x}
    &\Rightarrow (r_s r)^\frac{1}{2} \frac{\partial}{\partial r} = \frac{1}{2} \left( \frac{r_s r}{r_s^2} \frac{r_s}{r} \right)^\frac{1}{2} \frac{\partial}{\partial x}
                                                                = \frac{1}{2} \frac{\partial}{\partial x} \\
    &\hookrightarrow \frac{\partial S}{\partial t^\star} = \frac{3 x t_0}{2 r} \frac{\partial}{\partial r} \left( \frac{\partial}{\partial x} \left( \nu S \right) \right)
\end{align}

Puis la seconde :
\begin{align}
    \frac{3 x}{2 r} \frac{\partial}{\partial r} &= \frac{3}{4 r r_s} \frac{\partial}{\partial x} \\
    \hookrightarrow \frac{\partial S}{\partial t^\star} &= \frac{3 t_0}{4 r r_s} \frac{\partial^2 (\nu S)}{\partial x^2}
\end{align}

Enfin, en remplaçant $r$ d’après \eqref{eq:position_adim} :
\begin{equation}
    \frac{\partial S}{\partial t^\star} = \frac{3}{4} \frac{t_0}{r_s^2 x^2} \frac{\partial^2}{\partial x^2} \left(\nu S\right)
\end{equation}

Ce qui est bien l’\cref{eq:rel_densite_surface}.

\subsection{Vitesse radiale}

Nous partons de l’\cref{eq:vitesse_accretion} :
\begin{equation}
    v = - \frac{3}{\Sigma r^\frac{1}{2}} \frac{\partial}{\partial r} \left(\nu \Sigma r^\frac{1}{2} \right)
\end{equation}

Comme précédemment :
\begin{align}
    \Sigma &= \frac{S}{x} = S \left( \frac{r_s}{r} \right)^\frac{1}{2} \Rightarrow \Sigma r^\frac{1}{2} = S r_s^\frac{1}{2} \\
    \hookrightarrow v &= - \frac{3}{S r_s^\frac{1}{2}} \frac{\partial}{\partial r} \left(\nu S r_s^\frac{1}{2} \right)
\end{align}

Les $r_s^\frac{1}{2}$ se simplifient donc, de même que le font les $S_0$ en remplaçant $S = S^\star S_0$ :
\begin{equation}
    v = - \frac{3}{S^\star} \frac{\partial}{\partial r} \left(\nu S^\star \right)
\end{equation}

On utilise maintenant l’\cref{eq:dr_star} :
\begin{align}
    \frac{\partial}{\partial r} \left(\nu S^\star \right) &= \frac{1}{2 r_s x} \frac{\partial}{\partial x} \left( \nu S^\star \right) \\
    \hookrightarrow v &= - \frac{3}{2 r_s S^\star x} \frac{\partial}{\partial x} \left(\nu S^\star \right)
\end{align}

On remplace $\nu = \nu^\star \nu_0$ grâce à l’\cref{eq:viscosite_adim} :
\begin{align}
    \nu &= \frac{4}{3} \frac{r_s^2 \nu^\star}{t_0} \\
    \hookrightarrow v &= - \frac{2 r_s}{t_0 S^\star x} \frac{\partial}{\partial x} \left(\nu^\star S^\star \right)
\end{align}

Ce qui est bien l’\cref{eq:rel_vitesse_radiale}.

\subsection{Vitesse du son}

On commence par remplacer l’\cref{eq:rel_demi_hauteur_adim} dans l’\cref{eq:rel_viscosite_adim} :
\begin{equation}
    \nu^\star \Omega^\star = \alpha {c_s^\star}^2
\end{equation}

On fait de même avec l’\cref{eq:demi_hauteur} et l’\cref{eq:viscosite} :
\begin{equation}
    3 \nu \Omega = 2 \alpha c_s^2
\end{equation}

Et on divise cette dernière par la précédente :
\begin{equation}
    3 \frac{\nu}{\nu^\star} \frac{\Omega}{\Omega^\star} = 2 \left(\frac{c_s}{c_s^\star}\right)^2
\end{equation}

On remplace alors les rapports $\chi/\chi^\star$ par $\chi_0$ :
\begin{equation}
    3 \nu_0 \Omega_0 = 2 {c_s^0}^2
\end{equation}

On remplace alors $\nu_0 = \frac{4 r_s^2}{3 t_0}$, divise par \num{2} et prend la racine carrée :
\begin{equation}
    c_s^0 = \sqrt{2 \frac{r_s^2}{t_0} \Omega_0}
\end{equation}

Ce qui est bien l’\cref{eq:rel_cs_0}.

\subsection{Demi-hauteur}

\subsection{Évolution de la température}


