\section{Adimensionnement des variables et équations}

\subsection{Description physique du modèle}

Cette partie est fortement inspirée du document original fourni par Didier
\textsc{Pelat}, qui se base lui-même sur \citet{1984}.

Le modèle va dépendre de 6 paramètres d’entrée : $M$, $r_{max}$, $\dot{M_0}$, $\alpha$,
$X$ et $Y$. Ceux-ci sont décrits par la suite.

\subsubsection{Paramètres du trou noir}

Nous considérons un trou noir de Schwarzschild décrit par sa masse $M$, qui donne également son rayon de Schwarzschild~\eqref{eq:rayon_schwarzschild}.

\begin{equation}
    \label{eq:rayon_schwarzschild}
    r_s = \frac{2 G M}{c^2} \approx \SI{3.0}{\kilo\meter} \frac{M}{M_{\odot}}
\end{equation}

$G$ est la constante de gravitation, $c$ la vitesse de la lumière et $M_{\odot}$ la masse du soleil.

$r_min = 3 r_s$, $r_\mathrm{max} = 100 r_s$.

X, Y et Z composition chimique. On prend $X = \num{0.70}$, $Y = \num{0.28}$ et $Z = \num{0.02}$ (valeurs à la surface du soleil).

On s’intéresse à la description du disque à l’instant $t$, au voisinage d’un point $r$, décrit par sa température $T$, sa densité de surface $\Sigma$ et son taux d’accrétion local $\dot{M}$.

On va maintenant lister les équations régissant les différentes grandeurs applicables au disque. La plupart sont algébriques simples, deux sont des EDPs.

Vitesse angulaire : 
\begin{equation}
    \Omega = \left( \frac{G M}{r^3} \right)^½
\end{equation}

Masse atomique :
\begin{equation}
    \mu = \frac{1}{\left(2X + ¾Y + ½Z\right)} \approx 0.62
\end{equation}

Pression :
\begin{align}
    P &= P_{\mathrm{gaz}} + P_{\mathrm{rad}} \\
    P_{\mathrm{gaz}} &= \frac{\rho}{\mu m_p} k_B T \\
    P_{\mathrm{rad}} &= \frac{1}{3} a T^4
\end{align}

$\rho$ est la densité moyenne, $k_B$ la constante de Boltzmann, $m_p$ la masse du proton, $a$ la constante de radiation.

On définit l’indicateur de pression $\beta$ :
\begin{equation}
    \beta = \frac{P_{\mathrm{gaz}}}{P}
\end{equation}

La vitesse du son :
\begin{equation}
    c_s = \left( \frac{\Gamma_1 P}{\rho} \right)^½
\end{equation}

On prendra $\Gamma_1 = 1$ (expliquer).

La demi-hauteur du disque :
\begin{equation}
    H = \frac{c_s}{\Omega}
\end{equation}

La densité :
\begin{equation}
    \rho = \frac{\Sigma}{2 H}
\end{equation}

La viscosité :
\begin{equation}
    \nu = \frac{2}{3} \alpha c_s H
\end{equation}

Évolution de la densité de surface :
\begin{equation}
    \frac{\partial \Sigma}{\partial t} = \frac{3}{r} \frac{\partial}{\partial r} \left\{ r^½ \frac{\partial}{\partial r} \left(\nu \Sigma r^½ \right) \right\}
\end{equation}

Vitesse locale d’accrétion :
\begin{equation}
    v = − \frac{3}{\Sigma r^½} \frac{\partial}{\partial r} \left( \nu \Sigma r^½ \right)
\end{equation}

Taux d’accrétion :
\begin{equation}
    \dot{M} = − 2 \pi r \Sigma v
\end{equation}

Température :
\begin{equation}
    C_v \frac{\partial T}{\partial t} = \frac{9}{4} \Omega^2 − \frac{2 F_z}{\Sigma} + C_v \left[ (\Gamma_3 − 1) \frac{T}{\Sigma} \left( \frac{\partial \Sigma}{\partial t} + v \frac{\partial \Sigma}{\partial r}  \right) − v \frac{\partial T}{\partial r} \right]
\end{equation}

Où :
\begin{align}
    &C_v = \frac{R}{\mu} \frac{12 (\gamma_g − 1)(1 − \beta) + \beta}{(\gamma_g − 1) \beta} \\
    &C_v (\Gamma_3 − 1) = \frac{R}{\mu} \frac{4 − 3\beta}{\beta}
\end{align}

$\gamma_g$ est la capacité calorifique des gaz (\num{5/3} pour un gaz parfait monoatomique) et $R = k / m_p$.

Flux radiatif :
\begin{equation}
    F_z =
    \begin{cases}
        \frac{2 a c T^4}{3 (\kappa_\mathrm{ff} + \kappa_e)\Sigma}, &\text{si $\tau_\mathrm{eff} \geq 1$} \\
        \epsilon_\mathrm{ff} A H, &\text{si $\tau_\mathrm{eff} < 1$}
    \end{cases}
\end{equation}

Profondeur optique :
\begin{equation}
    \tau_\mathrm{eff} = ½ (\kappa_e \kappa_\mathrm{ff})^½ \Sigma
\end{equation}

Où :

\begin{align}
    \kappa_e &= \num{0.2} × (1+X) \si{\square\centi\meter\per\gram} \approx \SI{0.34}{\square\centi\meter\per\gram} \\
    \kappa_\mathrm{ff} &= \num{6.13e22} × \rho T^{-\frac{7}{2}} \si{\square\centi\meter\per\gram}\\
    \epsilon_\mathrm{ff} &= \num{6.22e20} × \rho^2 T^½ \\
    A &= 1
\end{align}

\subsection{Variables et équations adimensionnées}

Adimensionnement du temps :
\begin{equation}
    t^\star = t × \Omega_\mathrm{max}
\end{equation}

Où :
\begin{equation}
    \Omega_\mathrm{max} = \left( \frac{G M}{r^3_\mathrm{min}} \right)^½
\end{equation}

Adimensionnement de $r$ :
\begin{equation}
    x = \left( \frac{r}{r_s} \right)^½
\end{equation}

On introduit $S$ :
\begin{equation}
    S = \Sigma x
\end{equation}

Donc :
\begin{equation}
    \frac{\partial S}{\partial t^\star} = \frac{3}{4} \frac{1}{\Omega_\mathrm{max} r_s^2 x^2} \frac{\partial^2}{\partial x^2} \left(\nu S\right)
\end{equation}

On introduit $S^\star$, la grandeur adimensionnée correspondant à $S$ (qu’on définira plus tard puisque cela ne change rien ici) :
\begin{equation}
    \frac{\partial S^\star}{\partial t^\star} = \frac{3}{4} \frac{1}{\Omega_\mathrm{max} r_s^2 x^2} \frac{\partial^2}{\partial x^2} \left(\nu S^\star\right)
\end{equation}

On choisit donc :
\begin{equation}
    \nu^\star = \frac{3}{4} \frac{\nu}{\Omega_\mathrm{max} r_s^2} = \frac{\nu}{\nu_0}
\end{equation}

Ce qui donne :
\begin{equation}
    \frac{\partial S^\star}{\partial t^\star} = \frac{1}{x^2} \frac{\partial^2}{\partial x^2} \left(\nu^\star S^\star\right)
\end{equation}

On regarde les conséquences de ces choix pour l’expression de la vitesse radiale :
\begin{equation}
    v = − \frac{2 \Omega_\mathrm{max} r_s}{S^\star x} \frac{\partial}{\partial x} \left(\nu^\star S^\star\right)
\end{equation}

On pose donc :
\begin{equation}
    v_0 = \Omega_\mathrm{max} r_s \Rightarrow v^\star = \frac{v}{v_0} = − \frac{2}{S^\star x} \frac{\partial}{\partial x} \left(\nu^\star S^\star\right)
\end{equation}

On en profite pour adimensionner $c_s = \Omega H$ :
\begin{equation}
    c_s^\star = \frac{c_s}{v_0} = \frac{\Omega H}{\Omega_\mathrm{max} r_s}
\end{equation}

On pose donc logiquement :
\begin{equation}
    \Omega^\star = \frac{\Omega}{\Omega_\mathrm{max}} = \frac{3\sqrt{3}}{x^3}
\end{equation}

Et par conséquent :
\begin{equation}
    H^\star = \frac{H}{r_s}
\end{equation}

Ce qui nous donne bien :
\begin{equation}
    c_s^\star = \Omega^\star H^\star
\end{equation}

Pour le taux d’accrétion :
\begin{equation}
    \dot{M}^\star = \frac{\dot{M}}{\dot{M_0}} = − \frac{2 \pi x^2 r_s v^\star v_0 \Sigma}{\dot{M_0}} = − \frac{2 \pi r_s^2 \Omega_\mathrm{max} v^\star x^2 \Sigma}{\dot{M_0}} 
\end{equation}

Ce qui nous amène à poser :
\begin{equation}
    \Sigma^\star = \frac{\Sigma}{\Sigma_0} = \Sigma × \frac{2 \pi r_s^2 \Omega_\mathrm{max}}{\dot{M_0}}
\end{equation}

On en profite pour adimensionner naturellement $S$ :
\begin{equation}
    S^\star = \frac{S}{\Sigma_0} = \Sigma^\star x
\end{equation}

On obtient donc :
\begin{equation}
    \dot{M}^\star = v^\star S^\star x
\end{equation}

Pour la température, on pose :
\begin{equation}
    T^{\star} = \frac{T}{T_0}
\end{equation}

Où :
\begin{equation}
    T_0 = \left(\frac{1}{\sqrt{27}} × \frac{L_{tot}}{4 \pi r_s^2 \sigma} \right)^{\frac{1}{4}}
\end{equation}

Et :
\begin{equation}
    L_{tot} = \frac{1}{12} \dot{M_0} c^2
\end{equation}

Pour $F_z$ :
\begin{equation}
    F_z =
    \begin{cases}
        \frac{2 a c \left(T_0 T^\star\right)^4}{3 (\kappa_\mathrm{ff} + \kappa_e)\Sigma^\star \Sigma_0}, &\text{si $\tau_\mathrm{eff} \geq 1$} \\
        \epsilon_\mathrm{ff} r_s H^\star, &\text{si $\tau_\mathrm{eff} < 1$}
    \end{cases}
\end{equation}

Où :
\begin{align}
    \kappa_\mathrm{ff} &= \num{6.13e22} \rho_0 \rho^\star \left(T^\star T_0\right)^{-\frac{7}{2}} \si{\square\centi\meter\per\gram} \\
    \tau_\mathrm{eff} &= \frac{1}{2} \left[ \num{0.2} × (1 + X) × \num{6.13e22} \left(\rho_0 \rho^\star\right) \left(T^\star T_0\right)^{-\frac{7}{2}} \right]^½ \Sigma_0 \Sigma^\star \\
    \epsilon_\mathrm{eff} &= \num{6.22e20} (\rho_0 \rho^\star)^2 (T_0 T^\star)^½ \text{cgs}
\end{align}

Et :
\begin{equation}
    \rho^\star = \frac{\rho}{\rho_0} = \frac{\rho r_s}{\Sigma_0}
\end{equation}

Enfin :

\begin{equation}
    {P_\mathrm{gaz}}_0 = \frac{\rho_0}{\mu m_p} k_B T_0 \Rightarrow P_\mathrm{gaz}^\star = \frac{P_\mathrm{gaz}}{{P_\mathrm{gaz}}_0} = \rho^\star T^\star
\end{equation}

\begin{equation}
    {P_\mathrm{rad}}_0 = \frac{1}{3} a T_0^4 \Rightarrow P_\mathrm{rad}^\star = \frac{P_\mathrm{rad}}{{P_\mathrm{rad}}_0} = {T^\star}^4
\end{equation}

Ce qui donne pour $\beta$ :
\begin{equation}
    \beta = \frac{1}{1 + \frac{P_\mathrm{rad}^\star {P_\mathrm{rad}}_0}{P_\mathrm{gaz}^\star {P_\mathrm{gaz}}_0}} = \frac{1}{1 + \frac{{T^\star}^3}{\rho^\star} \frac{\mu m_p a T_0^3}{3 \rho_0 k_B}}
\end{equation}

On posera donc :
\begin{equation}
    B = \frac{\mu m_p a T_0^3}{3 \rho_0 k_B}
\end{equation}

Pour écrire :
\begin{equation}
    \beta = \frac{1}{1 + B \frac{{T^\star}^3}{\rho^\star}}
\end{equation}

La grande équation adimensionnée :

\begin{equation}
    C_v \frac{\partial T^{\star}}{\partial t^{\star}} =
    {\Omega^\star}^2 \frac{9 \Omega_\mathrm{max}}{4 T_0} − \frac{2 F_z x}{S^\star \Sigma_0 \Omega_\mathrm{max} T_0} +
    \frac{R}{\mu} \frac{4−3\beta}{\beta} \frac{T^\star}{S^\star} ×
    \left( \frac{\partial S^\star}{\partial t^\star} + \frac{v^\star}{2} \frac{\partial}{\partial x} \left(\frac{S^\star}{x}\right) \right) –
    \frac{C_v v^\star}{2 x} \frac{\partial T^\star}{\partial x}
\end{equation}

