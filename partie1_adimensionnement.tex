\section{Adimensionnement des variables et équations}

\subsection{Description physique du modèle}

Cette partie est très fortement inspirée du document original fourni par Didier
\textsc{Pelat}, qui se base lui-même sur \citet{1984}.

Le modèle va dépendre de 6 paramètres d’entrée : $M$, $r_{max}$, $\dot{M_0}$,
$\alpha$, $X$ et $Y$. Ceux-ci sont décrits par la suite.

Nous nous intéressons à la description du disque à l’instant $t$ et au
voisinage d’un point $r$ ; celui-ci est décrit par sa température $T$, sa
densité de surface $\Sigma$ et son taux d’accrétion local $\dot{M}$.
$\dot{M_0}$ représente le taux d’accrétion au bord du disque, c’est-à-dire la
quantité de matière apportée par le milieu extérieur par unité de temps.

Les équations régissant les différentes grandeurs applicables au disque sont
pour la plupart simplement algébriques, mais deux — celles donnant la variation
de $T$ et $\Sigma$ en fonction du temps — sont des EDPs.

\subsubsection{Paramètres du trou noir}

Nous considérons un trou noir de Schwarzschild décrit par sa masse $M$,
laquelle donne également son rayon de Schwarzschild :

\begin{equation}
    \label{eq:rayon_schwarzschild}
    r_s = \frac{\num{2}\, G M}{c^2} \approx \SI{3.0}{\kilo\meter} \frac{M}{M_{\odot}}
\end{equation}

Ici, $G = \SI{6.67408(31)d-8}{\cgs}$ est la
constante de gravitation, $c = \SI{2.99792458e10}{\cgs}$ la
vitesse de la lumière dans le vide et $M_{\odot} = \SI{1.98855(25)d33}{\cgs}$
la masse du soleil.

\subsubsection{Géométrie du disque}

Nous allons supposer que le disque d’accrétion est à symétrie cylindrique,
aussi nous ne regarderons que les équations faisant intervenir la coordonnée
radiale $r$. De plus, nous restreindrons notre étude sur une certaine plage de
valeur de ce paramètre. Pour la valeur minimale, la relativité générale stipule
qu’en-dessous de $r_\mathrm{min} = \num{3}\, r_s$ le disque est dynamiquement instable et ne
peut donc exister : ceci sera donc notre limite inférieure. Pour la valeur
maximale, il y a opposition entre deux problèmes : d’une part, les hypothèses
physiques que nous ferons par la suite doivent être vérifiées sur toute la
partie étudiée, d’autre part il faut que les instabilités qui se développeront
ne puisse pas atteindre la borne supérieure (elles se déplacent vers les $r$
croissants mais en diminuant d’amplitude) afin que les conditions prises pour
le bord extérieur par la restent valables. Pour choisir une bonne valeur, il
faudra en tester plusieurs numériquement, mais pour débuter et fixer un peu les
idées nous prendrons $r_\mathrm{max} = \num{100}\, r_s$.

\subsubsection{Composition chimique ($X$, $Y$ et $Z$)}

De manière usuelle, la composition chimique est représentée par les fractions
$X$, $Y$ et $Z$ (avec $X + Y + Z = 1$). $X$ représente la fraction en masse
d’hydrogène, $Y$ celle d’hélium et $Z$ celle de tous les autres éléments réunis
(et est appelée \textit{metallicité}). Nous prendrons les valeurs suivantes
(qui sont celles à la surface du soleil) :

\begin{equation}
    \label{eq:compo_chimique}
    X = \num{0.70}, Y = \num{0.28}, Z = \num{0.02}.
\end{equation}

\subsubsection{Vitesse angulaire $\Omega$}

La vitesse angulaire correspond à la période de l’orbite képlerienne à la distance $r$ :
 
\begin{equation}
    \label{eq:vitesse_angulaire}
    \Omega = \left( \frac{G M}{r^3} \right)^½
\end{equation}

\subsubsection{Masse atomique $\mu$}

Il s’agit de la masse moyenne des particules (noyaux et électrons) constituant
le gaz, rapportée à la masse atomique unitaire. Pour un gaz complètement ionisé
il est couramment utilisé :

\begin{equation}
    \label{eq:masse_atomique}
    \mu = \frac{1}{\left(2X + ¾Y + ½Z\right)} \approx 0.62
\end{equation}

Nous noterons au passage que cette valeur est constante et ne dépend que des
paramètres d’entrée $X$ et $Y$.

\subsubsection{Pressions $P$, $P_\mathrm{gaz}$ et $P_\mathrm{rad}$}

La pression se décompose en deux partie : une gazeuse due aux particules, une de radiation due aux photons.

\begin{align}
    \label{eq:pression}
    P &= P_{\mathrm{gaz}} + P_{\mathrm{rad}} \\
    P_{\mathrm{gaz}} &= \frac{\rho}{\mu m_p} k_B T \\
    P_{\mathrm{rad}} &= \frac{1}{3} a T^4
\end{align}

Ici, $\rho$ est la densité volumique moyenne de matière dans le disque, $k_B$
la constante de Boltzmann et $m_p$ la masse du proton avec $\frac{k_B}{m_p} =
\SI{8.31434e7}{\cgs}$, $a = \SI{7.564e-15}{\cgs}$ la constante de radiation.

Nous définissons au passage l’indicateur de pression $\beta$ :

\begin{equation}
    \label{eq:beta}
    \beta = \frac{P_{\mathrm{gaz}}}{P}
\end{equation}

Celui-ci vaut donc 1 si c’est la pression du gaz qui domine, 0 si c’est celle de radiation.

\subsubsection{Vitesse du son $c_s$}

Elle correspond à la vitesse de propagation des perturbations adiabatiques de densité.

\begin{equation}
    \label{eq:vitesse_son}
    c_s = \left( \frac{\Gamma_1 P}{\rho} \right)^½
\end{equation}

$\Gamma_1$ est le coefficient adiabatique, il dépend normalement de $\beta$.
Pour simplifier, nous prendront $\Gamma_1 = 1$ (techniquement, $\beta = \num{1}
\Rightarrow \Gamma_1 = \frac{5}{3}$ et $\beta = \num{0} \Rightarrow \Gamma_1 =
\frac{4}{3}$).

\subsubsection{Demi-hauteur du disque $H$}

L’équilibre hydrostatique donne :

\begin{equation}
    \label{eq:hauteur}
    H = \frac{c_s}{\Omega}
\end{equation}

\subsubsection{Densité volumique de matière $\rho$}

Par définition, sa valeur moyenne est directement :

\begin{equation}
    \label{eq:densite}
    \rho = \frac{\Sigma}{2 H}
\end{equation}

\subsubsection{Viscosité $\nu$}

La viscosité cinématique du gaz, d’origine turbulente, peut s’écrire :

\begin{equation}
    \label{eq:viscosite}
    \nu = \frac{2}{3} \alpha c_s H
\end{equation}

Ici, $\alpha$ est un paramètre \textit{ad-hoc} introduit par \citet{1973}. De
par sa nature phénoménologique, sa valeur n’est pas connue, mais cette forme
permet de rendre plutôt bien compte des phénomènes observés.

\subsubsection{Évolution de la densité de surface $\Sigma$}

Elle est donnée par une équation de type parabolique (« proche » d’une équation
de diffusion) :

\begin{equation}
    \label{eq:densite_surface}
    \frac{\partial \Sigma}{\partial t} = \frac{3}{r} \frac{\partial}{\partial r} \left\{ r^½ \frac{\partial}{\partial r} \left(\nu \Sigma r^½ \right) \right\}
\end{equation}

\subsubsection{Vitesse locale d’accrétion $v$}

Il s’agit de la composante radiale de la vitesse de la matière, c’est-à-dire
celle orientée vers le trou noir lorsque sa valeur est négative (\textit{i.e.}
quand la matière s’accrète).

\begin{equation}
    \label{eq:vitesse_accretion}
    v = - \frac{3}{\Sigma r^½} \frac{\partial}{\partial r} \left( \nu \Sigma r^½ \right)
\end{equation}

\subsubsection{Taux d’accrétion $\dot{M}$}

Il est local (car dépendant de r), il s’agit juste de prendre la quantité de
matière accrétée à la distance $r$ à chaque instant :

\begin{equation}
    \label{eq:taux_accretion}
    \dot{M} = - 2 \pi r \Sigma v
\end{equation}

\subsubsection{Évolution de la température \texorpdfstring{$T$}{\textit{T}}}

L’équation thermique s’écrit :

\begin{equation}
    \label{eq:equation_thermique}
    C_V \frac{\partial T}{\partial t} = Q^+ - Q^- + Q_\mathrm{adv}
\end{equation}

$C_V$ est la capacité calorifique massique à volume constant du mélange gaz et
radiation.

Les différents $Q$ correspondent à des termes d’échange de chaleur par unité de
masse et de temps.

Le terme de chauffage $Q^+$ est dû à la friction :

\begin{equation}
    \label{eq:chauffage}
    Q^+ = \frac{9}{4} \nu \Omega^2
\end{equation}

Le terme de dissipation $Q^-$ est dû au refroidissement par rayonnement :

\begin{align}
    Q^- &= 2 \frac{F_z}{\Sigma} + 2 \frac{H}{\Sigma r} \frac{\partial}{\partial r} \left(r F_r\right) \\
    \label{eq:refroidissement}
        &\approx 2 \frac{F_z}{\Sigma}
\end{align}

Ici, nous négligerons le flux radial (\textit{i.e.} selon le disque) $F_r$
devant le flux sortant du disque $F_z$.

Enfin, $Q_\mathrm{adv}$ est la 

\begin{equation}
    C_V \frac{\partial T}{\partial t} = \frac{9}{4} \nu \Omega^2 - \frac{2 F_z}{\Sigma} + C_V \left[ (\Gamma_3 - 1) \frac{T}{\Sigma} \left( \frac{\partial \Sigma}{\partial t} + v \frac{\partial \Sigma}{\partial r}  \right) - v \frac{\partial T}{\partial r} \right]
\end{equation}

Où :
\begin{align}
    &C_V = \frac{R}{\mu} \frac{12 (\gamma_g - 1)(1 - \beta) + \beta}{(\gamma_g - 1) \beta} \\
    &C_V (\Gamma_3 - 1) = \frac{R}{\mu} \frac{4 - 3\beta}{\beta}
\end{align}

$\gamma_g$ est la capacité calorifique des gaz (\num{5/3} pour un gaz parfait monoatomique) et $R = k / m_p$.

Flux radiatif :
\begin{equation}
    F_z =
    \begin{cases}
        \frac{2 a c T^4}{3 (\kappa_\mathrm{ff} + \kappa_e)\Sigma}, &\text{si $\tau_\mathrm{eff} \geq 1$} \\
        \epsilon_\mathrm{ff} A H, &\text{si $\tau_\mathrm{eff} < 1$}
    \end{cases}
\end{equation}

Profondeur optique :
\begin{equation}
    \tau_\mathrm{eff} = ½ (\kappa_e \kappa_\mathrm{ff})^½ \Sigma
\end{equation}

Où :

\begin{align}
    \kappa_e &= \num{0.2} × (1+X) \si{\square\centi\meter\per\gram} \approx \SI{0.34}{\square\centi\meter\per\gram} \\
    \kappa_\mathrm{ff} &= \num{6.13e22} × \rho T^{-\frac{7}{2}} \si{\square\centi\meter\per\gram}\\
    \epsilon_\mathrm{ff} &= \num{6.22e20} × \rho^2 T^½ \\
    A &= 1
\end{align}

\subsection{Variables et équations adimensionnées}

Adimensionnement du temps :
\begin{equation}
    t^\star = t × \Omega_\mathrm{max}
\end{equation}

Où :
\begin{equation}
    \Omega_\mathrm{max} = \left( \frac{G M}{r^3_\mathrm{min}} \right)^½
\end{equation}

Adimensionnement de $r$ :
\begin{equation}
    x = \left( \frac{r}{r_s} \right)^½
\end{equation}

On introduit $S$ :
\begin{equation}
    S = \Sigma x
\end{equation}

Donc :
\begin{equation}
    \frac{\partial S}{\partial t^\star} = \frac{3}{4} \frac{1}{\Omega_\mathrm{max} r_s^2 x^2} \frac{\partial^2}{\partial x^2} \left(\nu S\right)
\end{equation}

On introduit $S^\star$, la grandeur adimensionnée correspondant à $S$ (qu’on définira plus tard puisque cela ne change rien ici) :
\begin{equation}
    \frac{\partial S^\star}{\partial t^\star} = \frac{3}{4} \frac{1}{\Omega_\mathrm{max} r_s^2 x^2} \frac{\partial^2}{\partial x^2} \left(\nu S^\star\right)
\end{equation}

On choisit donc :
\begin{equation}
    \nu^\star = \frac{3}{4} \frac{\nu}{\Omega_\mathrm{max} r_s^2} = \frac{\nu}{\nu_0}
\end{equation}

Ce qui donne :
\begin{equation}
    \frac{\partial S^\star}{\partial t^\star} = \frac{1}{x^2} \frac{\partial^2}{\partial x^2} \left(\nu^\star S^\star\right)
\end{equation}

On regarde les conséquences de ces choix pour l’expression de la vitesse radiale :
\begin{equation}
    v = - \frac{2 \Omega_\mathrm{max} r_s}{S^\star x} \frac{\partial}{\partial x} \left(\nu^\star S^\star\right)
\end{equation}

On pose donc :
\begin{equation}
    v_0 = \Omega_\mathrm{max} r_s \Rightarrow v^\star = \frac{v}{v_0} = - \frac{2}{S^\star x} \frac{\partial}{\partial x} \left(\nu^\star S^\star\right)
\end{equation}

On en profite pour adimensionner $c_s = \Omega H$ :
\begin{equation}
    c_s^\star = \frac{c_s}{v_0} = \frac{\Omega H}{\Omega_\mathrm{max} r_s}
\end{equation}

On pose donc logiquement :
\begin{equation}
    \Omega^\star = \frac{\Omega}{\Omega_\mathrm{max}} = \frac{3\sqrt{3}}{x^3}
\end{equation}

Et par conséquent :
\begin{equation}
    H^\star = \frac{H}{r_s}
\end{equation}

Ce qui nous donne bien :
\begin{equation}
    c_s^\star = \Omega^\star H^\star
\end{equation}

Pour le taux d’accrétion :
\begin{equation}
    \dot{M}^\star = \frac{\dot{M}}{\dot{M_0}} = - \frac{2 \pi x^2 r_s v^\star v_0 \Sigma}{\dot{M_0}} = - \frac{2 \pi r_s^2 \Omega_\mathrm{max} v^\star x^2 \Sigma}{\dot{M_0}} 
\end{equation}

Ce qui nous amène à poser :
\begin{equation}
    \Sigma^\star = \frac{\Sigma}{\Sigma_0} = \Sigma × \frac{2 \pi r_s^2 \Omega_\mathrm{max}}{\dot{M_0}}
\end{equation}

On en profite pour adimensionner naturellement $S$ :
\begin{equation}
    S^\star = \frac{S}{\Sigma_0} = \Sigma^\star x
\end{equation}

On obtient donc :
\begin{equation}
    \dot{M}^\star = v^\star S^\star x
\end{equation}

Pour la température, on pose :
\begin{equation}
    T^{\star} = \frac{T}{T_0}
\end{equation}

Où :
\begin{equation}
    T_0 = \left(\frac{1}{\sqrt{27}} × \frac{L_{tot}}{4 \pi r_s^2 \sigma} \right)^{\frac{1}{4}}
\end{equation}

Et :
\begin{equation}
    L_{tot} = \frac{1}{12} \dot{M_0} c^2
\end{equation}

Pour $F_z$ :
\begin{equation}
    F_z =
    \begin{cases}
        \frac{2 a c \left(T_0 T^\star\right)^4}{3 (\kappa_\mathrm{ff} + \kappa_e)\Sigma^\star \Sigma_0}, &\text{si $\tau_\mathrm{eff} \geq 1$} \\
        \epsilon_\mathrm{ff} r_s H^\star, &\text{si $\tau_\mathrm{eff} < 1$}
    \end{cases}
\end{equation}

Où :
\begin{align}
    \kappa_\mathrm{ff} &= \num{6.13e22} \rho_0 \rho^\star \left(T^\star T_0\right)^{-\frac{7}{2}} \si{\square\centi\meter\per\gram} \\
    \tau_\mathrm{eff} &= \frac{1}{2} \left[ \num{0.2} × (1 + X) × \num{6.13e22} \left(\rho_0 \rho^\star\right) \left(T^\star T_0\right)^{-\frac{7}{2}} \right]^½ \Sigma_0 \Sigma^\star \\
    \epsilon_\mathrm{eff} &= \num{6.22e20} (\rho_0 \rho^\star)^2 (T_0 T^\star)^½ \text{cgs}
\end{align}

Et :
\begin{equation}
    \rho^\star = \frac{\rho}{\rho_0} = \frac{\rho r_s}{\Sigma_0}
\end{equation}

Enfin :

\begin{equation}
    {P_\mathrm{gaz}}_0 = \frac{\rho_0}{\mu m_p} k_B T_0 \Rightarrow P_\mathrm{gaz}^\star = \frac{P_\mathrm{gaz}}{{P_\mathrm{gaz}}_0} = \rho^\star T^\star
\end{equation}

\begin{equation}
    {P_\mathrm{rad}}_0 = \frac{1}{3} a T_0^4 \Rightarrow P_\mathrm{rad}^\star = \frac{P_\mathrm{rad}}{{P_\mathrm{rad}}_0} = {T^\star}^4
\end{equation}

Ce qui donne pour $\beta$ :
\begin{equation}
    \beta = \frac{1}{1 + \frac{P_\mathrm{rad}^\star {P_\mathrm{rad}}_0}{P_\mathrm{gaz}^\star {P_\mathrm{gaz}}_0}} = \frac{1}{1 + \frac{{T^\star}^3}{\rho^\star} \frac{\mu m_p a T_0^3}{3 \rho_0 k_B}}
\end{equation}

On posera donc :
\begin{equation}
    B = \frac{\mu m_p a T_0^3}{3 \rho_0 k_B}
\end{equation}

Pour écrire :
\begin{equation}
    \beta = \frac{1}{1 + B \frac{{T^\star}^3}{\rho^\star}}
\end{equation}

La grande équation adimensionnée :

\begin{equation}
    C_V \frac{\partial T^{\star}}{\partial t^{\star}} =
    {\Omega^\star}^2 \frac{3 \nu^\star v_0^2}{T_0} - \frac{2 F_z x}{S^\star \Sigma_0 \Omega_\mathrm{max} T_0} +
    \frac{R}{\mu} \frac{4-3\beta}{\beta} \frac{T^\star}{S^\star} ×
    \left( \frac{\partial S^\star}{\partial t^\star} + \frac{v^\star}{2} \frac{\partial}{\partial x} \left(\frac{S^\star}{x}\right) \right) –
    \frac{C_V v^\star}{2 x} \frac{\partial T^\star}{\partial x}
\end{equation}

