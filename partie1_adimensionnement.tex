\section{Adimensionnement des variables et équations}

\subsection{Description physique du modèle}

Cette partie est très fortement inspirée du document original fourni par Didier
\textsc{Pelat}, qui se base lui-même sur \citet{1984}.

Le modèle va dépendre de 6 paramètres d’entrée : $M$, $r_{max}$,
$\dot{M_0}$, $\alpha$, $X$ et $Y$. Ceux-ci sont décrits par la suite.

Nous nous intéressons à la description du disque à l’instant $t$ et au
voisinage d’un point $r$ ; celui-ci est décrit par sa température $T$, sa
densité de surface $\Sigma$ et son taux d’accrétion local $\dot{M}$.
$\dot{M_0}$ représente le taux d’accrétion au bord du disque, c’est-à-dire la
quantité de matière apportée par le milieu extérieur par unité de temps.

Les équations régissant les différentes grandeurs applicables au disque sont
pour la plupart simplement algébriques, mais deux — celles donnant la variation
de $T$ et $\Sigma$ en fonction du temps — sont des équations aux dérivées partielles (EDP).

\subsubsection{Paramètres du trou noir}

Nous considérons un trou noir de Schwarzschild décrit par sa masse $M$,
laquelle donne également son rayon de Schwarzschild :
\begin{equation}
    \label{eq:rayon_schwarzschild}
    r_s = \frac{2 G M}{c^2} \approx 3.0 \frac{M}{M_{\odot}} \si{\kilo\meter}
\end{equation}

Ici, $G = \SI{6.67408(31)d-8}{\cgs}$ est la constante de gravitation, $c =
\SI{2.99792458e10}{\cgs}$ la vitesse de la lumière dans le vide et $M_{\odot} =
\SI{1.98855(25)d33}{\cgs}$ la masse du soleil.

\subsubsection{Géométrie du disque}

Nous allons supposer que le disque d’accrétion est à symétrie cylindrique,
aussi nous ne regarderons que les équations faisant intervenir la coordonnée
radiale $r$. De plus, nous restreindrons notre étude sur une certaine plage de
valeur de ce paramètre. Pour la valeur minimale, la relativité générale stipule
qu’en-dessous de $r_\mathrm{min} = 3 \times r_s$ le disque est dynamiquement
instable et ne peut donc exister : ceci sera donc notre limite inférieure. Pour
la valeur maximale, il y a opposition entre deux problèmes : d’une part, les
hypothèses physiques que nous ferons par la suite doivent être vérifiées sur
toute la partie étudiée, d’autre part il faut que les instabilités qui se
développeront ne puisse pas atteindre la borne supérieure (elles se déplacent
vers les $r$ croissants mais en diminuant d’amplitude) afin que les conditions
prises pour le bord extérieur restent valables. Pour choisir une bonne
valeur, il faudra en tester plusieurs numériquement, mais pour débuter et fixer
un peu les idées nous prendrons $r_\mathrm{max} = 100 \times r_s$.

\subsubsection{\texorpdfstring{Composition chimique ($X$, $Y$ et $Z$)}{Composition chimique (X, Y et Z)}}

De manière usuelle, la composition chimique est représentée par les fractions
$X$, $Y$ et $Z$ (avec $X + Y + Z = 1$). $X$ représente la fraction en masse
d’hydrogène, $Y$ celle d’hélium et $Z$ celle de tous les autres éléments réunis
(et est appelée \textit{metallicité}). Nous prendrons les valeurs suivantes
(qui sont celles à la surface du Soleil) :
\begin{equation}
    \label{eq:compo_chimique}
    X = 0.70, Y = 0.28, Z = 0.02
\end{equation}

\subsubsection{\texorpdfstring{Vitesse angulaire $\Omega$}{Vitesse angulaire Ω}}

La vitesse angulaire correspond à la période de l’orbite képlerienne à la
distance $r$ :
\begin{equation}
    \label{eq:vitesse_angulaire}
    \Omega = \left( \frac{G M}{r^3} \right)^\frac{1}{2}
\end{equation}

Notons que c’est une constante du temps, nous pourrons donc la calculer une
seule fois en chaque position.

\subsubsection{\texorpdfstring{Masse atomique $\mu$}{Masse atomique μ}}

Il s’agit de la masse moyenne des particules (noyaux et électrons) constituant
le gaz, rapportée à la masse atomique unitaire. Pour un gaz complètement ionisé
il est couramment utilisé :
\begin{equation}
    \label{eq:masse_atomique}
    \mu = \frac{1}{2 X + \frac{3}{4} Y + \frac{1}{2} Z} \approx 0.62
\end{equation}

Notons au passage que cette valeur est constante et ne dépend que des
paramètres d’entrée $X$ et $Y$ donnés à l’\cref{eq:compo_chimique}.

\subsubsection{\texorpdfstring{Pressions $P$, $P_\mathrm{gaz}$ et $P_\mathrm{rad}$}{Pressions P, Pgaz et Prad}}

La pression se décompose en deux parties : une gazeuse due aux particules, une
de radiation due aux photons.
\begin{align}
    \label{eq:pression}
    P &= P_{\mathrm{gaz}} + P_{\mathrm{rad}} \\
    \label{eq:pression_gaz}
    P_{\mathrm{gaz}} &= \frac{\rho}{\mu m_p} k_B T \\
    \label{eq:pression_radiation}
    P_{\mathrm{rad}} &= \frac{1}{3} a T^4
\end{align}

Ici, $\rho$ est la densité volumique moyenne de matière dans le disque, $k_B$
la constante de Boltzmann et $m_p$ la masse du proton avec $\frac{k_B}{m_p} =
\SI{8.3144598(48)d7}{\cgs}$, $a = \frac{8 \pi^5 k_B^4}{15 h^3 c^3} =
\SI{7.5657308531642009e-17}{\cgs}$ la constante de radiation ($h$ est la
constante de Planck).

Nous définissons au passage l’indicateur de pression $\beta$ :
\begin{equation}
    \label{eq:indicateur_pression}
    \beta = \frac{P_{\mathrm{gaz}}}{P}
\end{equation}

Celui-ci vaut donc 1 si c’est la pression du gaz qui domine, 0 si c’est celle
de radiation.

\subsubsection{\texorpdfstring{Vitesse du son $c_s$}{Vitesse du son cs}}

Elle correspond à la vitesse de propagation des perturbations adiabatiques de
densité.
\begin{equation}
    \label{eq:vitesse_son}
    c_s = \left( \frac{\Gamma_1 P}{\rho} \right)^\frac{1}{2}
\end{equation}

$\Gamma_1$ est le coefficient adiabatique, il dépend normalement de $\beta$.
Pour simplifier, nous prendrons $\Gamma_1 = 1$ (techniquement, $\beta = 1
\Rightarrow \Gamma_1 = \frac{5}{3}$ et $\beta = 0 \Rightarrow \Gamma_1
= \frac{4}{3}$).

\subsubsection{\texorpdfstring{Demi-hauteur du disque $H$}{Demi-hauteur du disque H}}

L’équilibre hydrostatique donne :
\begin{equation}
    \label{eq:demi_hauteur}
    H = \frac{c_s}{\Omega}
\end{equation}

\subsubsection{\texorpdfstring{Densité volumique de matière $\rho$}{Densité volumique de matière ρ}}

Par définition, sa valeur moyenne est directement :
\begin{equation}
    \label{eq:densite_volumique}
    \rho = \frac{\Sigma}{2 H}
\end{equation}

\subsubsection{\texorpdfstring{Viscosité $\nu$}{Viscosité ν}}

La viscosité cinématique du gaz, d’origine turbulente, peut s’écrire dans le modèle de disque $\alpha$:
\begin{equation}
    \label{eq:viscosite}
    \nu = \frac{2}{3} \alpha c_s H
\end{equation}

Ici, $\alpha$ est un paramètre \textit{ad-hoc} introduit par \citet{1973}. De
par sa nature phénoménologique, sa valeur n’est pas connue, mais cette forme
permet de bien rendre compte des phénomènes observés.

\subsubsection{\texorpdfstring{Évolution de la densité de surface $\Sigma$}{Évolution de la densité de surface Σ}}

Elle est donnée par une équation de type parabolique (« proche » d’une équation
de diffusion) :
\begin{equation}
    \label{eq:densite_surface}
    \frac{\partial \Sigma}{\partial t} = \frac{3}{r} \frac{\partial}{\partial r} \left\{ r^\frac{1}{2} \frac{\partial}{\partial r} \left(\nu \Sigma r^\frac{1}{2} \right) \right\}
\end{equation}

\subsubsection{\texorpdfstring{Vitesse locale d’accrétion $v$}{Vitesse locale d’accrétion v}}

Il s’agit de la composante radiale de la vitesse de la matière, c’est-à-dire
celle orientée vers le trou noir lorsque sa valeur est négative (\textit{i.e.}
quand la matière s’accrète).
\begin{equation}
    \label{eq:vitesse_accretion}
    v = - \frac{3}{\Sigma r^\frac{1}{2}} \frac{\partial}{\partial r} \left( \nu \Sigma r^\frac{1}{2} \right)
\end{equation}

\subsubsection{\texorpdfstring{Taux d’accrétion $\dot{M}$}{Taux d’accrétion dM/dt}}

Il est local (car dépendant de r), il s’agit juste de prendre la quantité de
matière accrétée à la distance $r$ à chaque instant :
\begin{equation}
    \label{eq:taux_accretion}
    \dot{M} = - 2 \pi r \Sigma v
\end{equation}

\subsubsection{\texorpdfstring{Évolution de la température $T$}{Évolution de la température \textit{T}}}

L’équation thermique s’écrit :
\begin{equation}
    \label{eq:equation_thermique}
    C_v \frac{\partial T}{\partial t} = Q^+ - Q^- + Q_\mathrm{adv}
\end{equation}

$C_v$ est la capacité calorifique massique à volume constant du mélange gaz et
radiation.

Les différents $Q$ correspondent à des termes d’échange de chaleur par unité de
masse et de temps.

Le terme de chauffage $Q^+$ est dû à la friction :
\begin{equation}
    \label{eq:chauffage}
    Q^+ = \frac{9}{4} \nu \Omega^2
\end{equation}

Le terme de dissipation $Q^-$ est dû au refroidissement par rayonnement :
\begin{align}
    Q^- &= 2 \frac{F_z}{\Sigma} + 2 \frac{H}{\Sigma r} \frac{\partial}{\partial r} \left(r F_r\right) \\
    \label{eq:refroidissement}
        &\approx 2 \frac{F_z}{\Sigma}
\end{align}

Ici, nous négligerons le flux radial (\textit{i.e.} selon le disque) $F_r$
devant le flux sortant du disque $F_z$.

Enfin, $Q_\mathrm{adv}$ est la chaleur transportée par la matière en mouvement
(terme d’advection) :
\begin{equation}
    \label{eq:advection}
    Q_\mathrm{adv} = C_v \left[ (\Gamma_{3} - 1) \frac{T}{\Sigma} \left( \frac{\partial \Sigma}{\partial t} + v \frac{\partial \Sigma}{\partial r}  \right) - v \frac{\partial T}{\partial r} \right]
\end{equation}

Dans les \cref{eq:equation_thermique,eq:advection} nous avons utilisé $C_v$ et
$\Gamma_3$, le sens physique du premier a déjà été donné, le second est un
exposant adiabatique. Leur expression :
\begin{align}
    \label{eq:capacite_calorifique}
    &C_v = \frac{\mathcal{R}}{\mu} \frac{12 (\gamma_g - 1)(1 - \beta) + \beta}{(\gamma_g - 1) \beta} \\
    &C_v (\Gamma_{3} - 1) = \frac{\mathcal{R}}{\mu} \frac{4 - 3\beta}{\beta}
\end{align}

$\gamma_g$ est le rapport des capacités calorifiques des gaz (5/3 pour un gaz
parfait monoatomique) et nous avons introduit pour simplifier la constante des
gaz parfaits $\mathcal{R} = \frac{k_B}{m_p} = \SI{8.3144598(48)d7}{\cgs}$.

En remplaçant les \cref{eq:chauffage,eq:refroidissement,eq:advection} dans
l’\cref{eq:equation_thermique} nous obtenons :
\begin{equation}
    \label{eq:temperature}
    C_v \frac{\partial T}{\partial t} = \frac{9}{4} \nu \Omega^2 - \frac{2 F_z}{\Sigma} + C_v \left[ (\Gamma_{3} - 1) \frac{T}{\Sigma} \left( \frac{\partial \Sigma}{\partial t} + v \frac{\partial \Sigma}{\partial r} \right) - v \frac{\partial T}{\partial r} \right]
\end{equation}

\subsubsection{\texorpdfstring{Flux radiatif $F_z$}{Flux radiatif Fz}}

Il reste donc à définir le flux radiatif $F_z$ qui intervient dans les
\cref{eq:refroidissement,eq:equation_thermique} :
\begin{equation}
    \label{eq:flux_radiatif}
    F_z =
    \begin{cases}
        \frac{2 a c T^4}{3 (\kappa_\mathrm{ff} + \kappa_e)\Sigma}, &\text{si $\tau_\mathrm{eff} \geq 1$} \\
        \epsilon_\mathrm{ff} A H, &\text{si $\tau_\mathrm{eff} < 1$}
    \end{cases}
\end{equation}

Dont l’expression dépend donc de la profondeur optique $\tau_\mathrm{eff}$ :
\begin{equation}
    \label{eq:tau_eff}
    \tau_\mathrm{eff} = \frac{1}{2} (\kappa_e \kappa_\mathrm{ff})^\frac{1}{2} \Sigma
\end{equation}

Et où nous avons introduit les grandeurs suivantes :
\begin{subequations}
    \begin{align}
        \label{eq:opacite_thomson}
        \kappa_e &= 0.2 \times (1 + X) \si{\square\centi\meter\per\gram} \approx \SI{0.34}{\square\centi\meter\per\gram} \\
        \label{eq:opacite_bremsstrahlung}
        \kappa_\mathrm{ff} &= \num{6.13e22} \times \rho T^{-\frac{7}{2}} \si{\square\centi\meter\per\gram}\\
        \label{eq:emissivite}
        \epsilon_\mathrm{ff} &= \num{6.22e20} \times \rho^2 T^\frac{1}{2} \si{\cgs} \\
        A &= 1
    \end{align}
\end{subequations}

$\kappa_e$ est l’opacité correspondant à la diffusion Thomson,
$\kappa_\mathrm{ff}$ est celle associée au bremsstrahlung (rayonnement de
freinage), $\epsilon_\mathrm{ff}$ est l’émissivité associée à ce phénomène et
$A$ et le coefficient d’amplification Compton (la valeur prise ici est choisie
pour simplifier).

\subsection{Variables et équations adimensionnées}

Il est intéressant d’adimensionner toutes les grandeurs en jeu pour plusieurs
raisons. Notamment, celà permet de simplifier l’écriture des équations ou
encore d’éviter un certain nombre de problèmes numériques liés à la
représentation des nombres en informatique.

Le fil directeur suivi ici est d’obtenir des adimensionnements naturels et
ayant un sens physique, mais avec la contrainte première d’obtenir des
équations du système ne faisant pas intervenir de coefficients numériques dans
la mesure du possible.

Dans toute la suite, nous noterons $\chi^\star$ la grandeur adimensionnée
correspondant à la grandeur physique $\chi$.

Les calculs, peu abordés ici, sont détaillés dans
l’\cref{app:adimensionnement}.

\subsubsection{\texorpdfstring{Temps $t^\star$}{Temps t⋆}}

Nous commençons par adimensionner la variable temporelle :
\begin{equation}
    \label{eq:temps_adim}
    t^\star = \frac{t}{t_0}
\end{equation}

Où $t_0$ est une grandeur homogène à un temps qui sera définie ultérieurement.
Nous introduisons ici cette notation pour pouvoir réécrire les différentielles
qui vont intervenir dans les différentes équations.

\subsubsection{\texorpdfstring{Position $x$}{Position x}}

L’autre variable indépendante est la position $r$. Nous l’adimensionnons
naturellement en fonction du rayon de Schwarzschild $r_s$. En revanche, les
termes en $r^\frac{1}{2}$ apparaissant dans certaines équations comme
l’\cref{eq:vitesse_accretion} donnent envie de définir la variable
adimensionnée correspondant à $r$ ainsi :
\begin{equation}
    \label{eq:position_adim}
    x = \left(\frac{r}{r_s}\right)^\frac{1}{2}
\end{equation}

Nous ne la noterons pas $r^\star$ pour justement marquer cette spécificité. Ce
choix a de l’intérêt pour la simplification de l’\cref{eq:densite_surface}
également.

\subsubsection{\texorpdfstring{Évolution de la densité de surface $S^\star$, viscosité $\nu^\star$}{Évolution de la densité de surface S⋆, viscosité ν⋆}}

En effet, nous définissons $S$ (qu’on appellera par la suite densité de
surface — presque sans abus de language) :
\begin{equation}
    \label{eq:definition_S}
    S = \Sigma x
\end{equation}

Ce qui nous donne en réintroduisant dans l’\cref{eq:densite_surface} (détails
des calculs dans l’\cref{app:densite_surface}) :
\begin{equation}
    \label{eq:rel_densite_surface}
    \frac{\partial S}{\partial t^\star} = \frac{3}{4} \frac{t_0}{r_s^2 x^2} \frac{\partial^2}{\partial x^2} \left(\nu S\right)
\end{equation}

Nous introduisons maintenant $S^\star$, la grandeur adimensionnée correspondant
à $S$ :
\begin{equation}
    \label{eq:S_adim}
    S^\star = \frac{S}{S_0}
\end{equation}

$S_0$ est une grandeur homogène à $S$ (que nous définirons plus tard puisque
cela ne change rien ici). En remplaçant dans l’équation précédente :
\begin{equation}
    \frac{\partial S^\star}{\partial t^\star} = \frac{3}{4} \frac{t_0}{r_s^2 x^2} \frac{\partial^2}{\partial x^2} \left(\nu S^\star\right)
\end{equation}

Nous posons donc :
\begin{equation}
    \label{eq:viscosite_adim}
    \nu^\star = \frac{3}{4} \frac{t_0 \nu}{r_s^2} = \frac{\nu}{\nu_0} \Rightarrow \nu_0 = \frac{4}{3} \frac{r_s^2}{t_0}
\end{equation}

Et nous obtenons :
\begin{equation}
    \label{eq:rel_densite_surface_adim}
    \frac{\partial S^\star}{\partial t^\star} = \frac{1}{x^2} \frac{\partial^2}{\partial x^2} \left(\nu^\star S^\star\right)
\end{equation}

\subsubsection{\texorpdfstring{Vitesse locale d’accrétion $v^\star$}{Vitesse locale d’accrétion v⋆}}

Ces choix ont des conséquences pour l’expression de la vitesse d’accrétion
(aussi dite radiale) définie par l’\cref{eq:vitesse_accretion} (détails des
calculs dans l’\cref{app:vitesse_accretion}) :
\begin{equation}
    \label{eq:rel_vitesse_accretion}
    v = - \frac{2 r_s}{t_0 S^\star x} \frac{\partial}{\partial x} \left(\nu^\star S^\star\right)
\end{equation}

Ce qui nous amène à poser :
\begin{equation}
    \label{eq:vitesse_accretion_adim}
    v^\star = \frac{v}{v_0} = \frac{v t_0}{2 r_s} \Rightarrow v_0 = 2 \frac{r_s}{t_0}
\end{equation}

Nous obtenons donc :
\begin{equation}
    \label{eq:rel_vitesse_accretion_adim}
    v^\star = - \frac{1}{S^\star x} \frac{\partial}{\partial x} \left(\nu^\star S^\star\right)
\end{equation}

\subsubsection{\texorpdfstring{Vitesse angulaire $\Omega^\star$}{Vitesse angulaire Ω⋆}}

$\Omega$ étant une constante du temps et ne dépendant que de la position $r$,
il est souhaitable d’en obtenir un adimensionnement correspondant
$\Omega^\star$ aussi simple que possible en fonction de $x$. En combinant les
\cref{eq:vitesse_angulaire,eq:position_adim} nous obtenons :
\begin{equation}
    \label{eq:rel_omega}
    \Omega = \frac{1}{x^3} \left( \frac{G M}{r_s^3} \right)^\frac{1}{2}
\end{equation}

Nous voudrions donc avoir :
\begin{equation}
    \label{eq:rel_omega_adim}
    \Omega^\star = \frac{1}{x^3}
\end{equation}

Ce qui nous amène naturellement à poser :
\begin{equation}
    \label{eq:vitesse_angulaire_adim}
    \Omega^{\star} = \frac{\Omega}{\Omega_0} = \Omega \left( \frac{r_s^3}{G M} \right)^\frac{1}{2} \Rightarrow \Omega_0 = \left( \frac{G M}{r_s^3} \right)^{\frac{1}{2}}
\end{equation}

\subsubsection{\texorpdfstring{Vitesse du son $c_s^\star$}{Vitesse du son cs⋆}}

Naturellement, nous introduisons $c_s^\star$ :
\begin{equation}
   \label{eq:vitesse_son_adim}
   c_s^\star = \frac{c_s}{c_s^0}
\end{equation}

Toujours dans l’idée de simplifier au maximum les relations entre les
différentes grandeurs adimensionnées, nous souhaiterions que
l’\cref{eq:viscosite} se réécrive ainsi :
\begin{equation}
    \label{eq:rel_viscosite_adim}
    \nu^\star = \alpha c_s^\star H^\star
\end{equation}

Nous gardons le coefficient $\alpha$, car il définit cette loi et n’est pas à
proprement parler un coefficient numérique (mais de toutes façons, nous le
prendrons – au moins dans un premier temps — comme égal à 1 dans les
simulations). Nous n’avons pas encore défini $H^\star$, mais nous pouvons le
substituer par des grandeurs définies en imposant que, de même,
l’\cref{eq:demi_hauteur} se réécrive ainsi :
\begin{equation}
    \label{eq:rel_demi_hauteur_adim}
    H^\star = \frac{c_s^\star}{\Omega^\star}
\end{equation}

En combinant ces deux conditions avec les \cref{eq:demi_hauteur,eq:viscosite} et les adimensionnements
établis aux \cref{eq:viscosite_adim,eq:vitesse_angulaire_adim} nous obtenons :
\begin{equation}
    \label{eq:rel_vitesse_son_zero}
    c_s^0 = \sqrt{2 \frac{r_s^2}{t_0} \Omega_0}
\end{equation}

À partir d’ici, il faut faire un choix sur l’adimensionnement de $t$ et $c_s$.
En effet, ce sont les seules grandeurs encore non-définies dans cette équation.
Il y a pragmatiquement deux possibilités pour obtenir des expressions simples :
soit définir $t_0$ comme l’inverse d’$\Omega_0$, soit poser ${c_s}_0 = v_0$.
Ici, nous choisirons cette seconde possibilité :
\begin{equation}
    \label{eq:vitesse_son_zero}
    c_s^0 = v_0 = 2 \frac{r_s}{t_0}
\end{equation}

Ce qui pour $t_0$ nous donne :
\begin{equation}
    \label{eq:temps_zero}
    t_0 = \frac{2}{\Omega_0} = 2 \left( \frac{r_s^3}{G M} \right)^\frac{1}{2}
\end{equation}

\subsubsection{\texorpdfstring{Demi-hauteur $H^\star$}{Demi-hauteur H⋆}}

Commençons par définir $H^\star$ :
\begin{equation}
    \label{eq:demi_hauteur_adim}
    H^\star = \frac{H}{H_0}
\end{equation}

En réinjectant cela et l’\cref{eq:vitesse_son_adim} dans
l’\cref{eq:rel_demi_hauteur_adim} nous obtenons :
\begin{equation}
    \label{eq:demi_hauteur_zero}
    H_0 = \frac{c_s^0}{\Omega_0} = r_s
\end{equation}

\subsubsection{\texorpdfstring{Taux d’accrétion $\dot{M}^\star$, densité de surface $\Sigma^\star$}{Taux d’accrétion (dM/dt)⋆, densité de surface Σ⋆}}

En partant de l’\cref{eq:taux_accretion} :
\begin{equation}
    \label{eq:rel_taux_accretion}
    \dot{M}^\star = \frac{\dot{M}}{\dot{M_0}} = - \frac{2 \pi x^2 r_s \Sigma v^\star v_0}{\dot{M_0}} = - \frac{4 \pi r_s^2 x^2 \Sigma v^\star}{\dot{t_0 M_0}}
\end{equation}

Ce qui nous amène à poser :
\begin{equation}
    \label{eq:densite_surface_adim}
    \Sigma^\star = \frac{\Sigma}{\Sigma_0} = \Sigma \times \frac{4 \pi r_s^2}{t_0 \dot{M_0}} = \Sigma \times \frac{3 \pi \nu_0}{\dot{M_0}} \Rightarrow \Sigma_0 = \frac{\dot{M_0}}{3 \pi \nu_0}
\end{equation}

Celà nous permet au passage d’adimensionner naturellement $S$ :
\begin{equation}
    \label{eq:S_zero}
    S^\star = \frac{S}{S_0} = \frac{\Sigma x}{\Sigma_0} = \Sigma^\star x \Rightarrow S_0 = \Sigma_0
\end{equation}

Et donc (qui était en fait le résultat voulu) :
\begin{equation}
    \label{eq:rel_taux_accretion_adim}
    \dot{M}^\star = - x S^\star v^\star
\end{equation}

\subsubsection{\texorpdfstring{Densité volumique $\rho^\star$}{Densité volumique ρ⋆}}

Nous souhaiterions obtenir pour l’\cref{eq:densite_volumique} :
\begin{equation}
    \label{eq:rel_densite_volumique_adim}
    \rho^\star = \frac{\Sigma^\star}{H^\star}
\end{equation}

Étant donné les \cref{eq:demi_hauteur_adim,eq:densite_surface_adim} cela nous
impose :
\begin{equation}
    \label{eq:densite_volumique_adim}
    \rho^\star = \frac{\rho}{\rho_0} = \frac{2 H_0 \rho}{\Sigma_0} \Rightarrow \rho_0 = \frac{\Sigma_0}{2 H_0} = \frac{\Sigma_0}{2 r_s}
\end{equation}

\subsubsection{\texorpdfstring{Température $T^\star$, évolution et flux radiatif $F_z^\star$}{Température T⋆, évolution et flux radiatif Fz⋆}}

Les équations faisant intervenir un terme en $T^4$, le choix de cet
adimensionnement est particulièrement critique. Il est suggéré un argument
énergétique pour estimer un ordre de grandeur de la température.
%Celui-ci et les calculs correspondant sont présentés à l’\cref{app:temperature}.

Comme d’habitude :
\begin{equation}
    \label{eq:temperature_adim}
    T^{\star} = \frac{T}{T_0}
\end{equation}

Où :
\begin{equation}
    \label{eq:temperature_zero}
    T_0 = \left(\frac{1}{\sqrt{27}} \times \frac{L_{tot}}{4 \pi r_s^2 \sigma} \right)^\frac{1}{4}
\end{equation}

Nous avons introduit $\sigma = \frac{a c}{4}$ (il s’agit de la constante de
Stefan–Boltzmann) et :
\begin{equation}
    L_{tot} = \frac{1}{12} \dot{M_0} c^2
\end{equation}

Nous réinjectons tous ces adimensionnements dans l’\cref{eq:temperature} :
%(les calculs sont dans l’\cref{app:temperature}):
\begin{equation}
    \label{eq:rel_temperature}
    C_v T_0 \frac{\partial T^{\star}}{\partial t^{\star}} =
    3 v_0^2 \nu^\star {\Omega^\star}^2 - \frac{2 t_0 F_z x}{S^\star \Sigma_0} +
    \frac{\mathcal{R} T_0}{\mu} \frac{4-3\beta}{\beta} \frac{T^\star}{S^\star} \times
    \left( \frac{\partial S^\star}{\partial t^\star} + v^\star \frac{\partial}{\partial x} \left(\frac{S^\star}{x}\right) \right) -
    \frac{C_v T_0 v^\star}{x} \frac{\partial T^\star}{\partial x}
\end{equation}

Naturellement, nous avons envie d’introduire $F_z^\star$ pour simplifier cette
écriture. Nous poserons :
\begin{equation}
    \label{eq:flux_radiatif_adim}
    F_z^\star = \frac{F_z}{F_z^0} = \frac{2 t_0 F_z}{\Sigma_0} \Rightarrow F_z^0 = \frac{\Sigma_0}{2 t_0} = \frac{\rho_0 H_0}{t_0}
\end{equation}

Cela ne donne pas à $F_0$ la même dimension que $F$ (ce qui serait difficile à
obtenir et pas forcément souhaitable), mais permet de simplifier grandement les
choses. En remplaçant ceci dans l’\cref{eq:rel_temperature} nous avons donc :
\begin{equation}
    C_v T_0 \frac{\partial T^{\star}}{\partial t^{\star}} =
    3 v_0^2 \nu^\star {\Omega^\star}^2 - \frac{F_z^\star x}{S^\star} +
    \frac{\mathcal{R} T_0}{\mu} \frac{4-3\beta}{\beta} \frac{T^\star}{S^\star} \times
    \left( \frac{\partial S^\star}{\partial t^\star} + v^\star \frac{\partial}{\partial x} \left(\frac{S^\star}{x}\right) \right) -
    \frac{C_v T_0 v^\star}{x} \frac{\partial T^\star}{\partial x}
\end{equation}

Il ne reste plus qu’à simplifier l’expression de $F_z^\star$ par rapport à
celle de $F_z$~\eqref{eq:flux_radiatif} :
\begin{equation}
    F_z^\star =
    \begin{cases}
        \frac{2 c^2 x {T^\star}^4}{27 \sqrt{3} (\kappa_\mathrm{ff} + \kappa_e) S^\star \Sigma_0}, &\text{si $\tau_\mathrm{eff} \geq 1$} \\
        \frac{t_0}{\rho_0}\epsilon_\mathrm{ff} H^\star, &\text{si $\tau_\mathrm{eff} < 1$}
    \end{cases}
\end{equation}

Où :
\begin{subequations}
    \begin{align}
        \kappa_\mathrm{ff} &= \num{6.13e22} (\rho_0 \rho^\star) {T_0 T^\star}^{-\frac{7}{2}} \si{\square\centi\meter\per\gram} \\
        \kappa_e &= 0.2 \times (1 + X) \si{\square\centi\meter\per\gram} \approx \SI{0.34}{\square\centi\meter\per\gram} \\
        \epsilon_\mathrm{ff} &= \num{6.22e20} \times \rho^2 T^\frac{1}{2} \si{\cgs} \\
        \tau_\mathrm{eff} &= \frac{1}{2} (\kappa_e \kappa_\mathrm{ff})^\frac{1}{2} \frac{S^\star}{x} \Sigma_0
    \end{align}
\end{subequations}

Enfin nous introduisons $C_v^\star$ :
\begin{equation}
    \label{eq:capacite_calorifique_adim}
    C_v^\star = \frac{C_v}{C_v^0} = C_v T_0 \Rightarrow C_v^0 = T_0^{-1}
\end{equation}

Ce qui nous donne finalement :
\begin{equation}
    \label{eq:rel_temperature_adim}
    C_v^\star \frac{\partial T^{\star}}{\partial t^{\star}} =
    3 v_0^2 \nu^\star {\Omega^\star}^2 - \frac{F_z^\star x}{S^\star} +
    \frac{\mathcal{R} T_0}{\mu} \frac{4-3\beta}{\beta} \frac{T^\star}{S^\star} \times
    \left( \frac{\partial S^\star}{\partial t^\star} + v^\star \frac{\partial}{\partial x} \left(\frac{S^\star}{x}\right) \right) -
    \frac{C_v^\star v^\star}{x} \frac{\partial T^\star}{\partial x}
\end{equation}

Et en combinant les \cref{eq:capacite_calorifique,eq:capacite_calorifique_adim} :
\begin{equation}
    \label{eq:rel_capacite_calorifique_adim}
    C_v^\star = \frac{\mathcal{R} T_0}{\mu} \frac{12 (\gamma_g - 1)(1 - \beta) + \beta}{(\gamma_g - 1) \beta}
\end{equation}

\subsubsection{\texorpdfstring{Pressions $P_\mathrm{gaz}^\star$ et $P_\mathrm{rad}^\star$, indicateur $\beta$}{Pressions Pgaz⋆ et Prad⋆, β}}

Pour les deux termes de pression, nous faisons en sorte d’avoir les expressions
les plus simples possibles.

Pour le gaz, réduisons l’\cref{eq:pression_gaz} à l’essentiel :
\begin{equation}
    \label{eq:rel_pression_gaz_adim}
    P_\mathrm{gaz}^\star = \rho^\star T^\star
\end{equation}

Ce qui implique :
\begin{equation}
    \label{eq:pression_gaz_adim}
    P_\mathrm{gaz}^\star = \frac{P_\mathrm{gaz}}{P_\mathrm{gaz}^0} = \frac{\mu m_p}{\rho_0 k_B T_0} P_\mathrm{gaz} \Rightarrow P_\mathrm{gaz}^0 = \frac{\rho_0 k_B T_0}{\mu m_p} = \rho_0 \frac{\mathcal{R} T_0}{\mu}
\end{equation}

Idem pour l’\cref{eq:pression_radiation} décrivant la radiation :
\begin{equation}
    \label{eq:rel_pression_radiation_adim}
    P_\mathrm{rad}^\star = {T^\star}^4
\end{equation}

Soit :
\begin{equation}
    \label{eq:pression_radiation_adim}
    P_\mathrm{rad}^\star = \frac{P_\mathrm{rad}}{P_\mathrm{rad}^0} = \frac{3}{a T_0^4} P_\mathrm{rad} \Rightarrow P_\mathrm{rad}^0 = \frac{1}{3} a T_0^4
\end{equation}

Cela donne pour $\beta$ (par rapport à l’\cref{eq:indicateur_pression}, nous
avons simplement simplifié par $P_\mathrm{gaz}$) :

\begin{equation}
    \label{eq:rel_indicateur_pression}
    \beta = \frac{1}{1 + \frac{P_\mathrm{rad}^\star P_\mathrm{rad}^0}{P_\mathrm{gaz}^\star P_\mathrm{gaz}^0}} = \frac{1}{1 + \frac{{T^\star}^3}{\rho^\star} \frac{\mu m_p a T_0^3}{3 \rho_0 k_B}}
\end{equation}

Nous poserons donc :
\begin{equation}
    \label{eq:indicateur_pression_zero}
    \beta_0 = \frac{P_\mathrm{rad}^0}{P_\mathrm{gaz}^0} = \frac{\mu m_p a T_0^3}{3 \rho_0 k_B}
\end{equation}

Pour écrire :
\begin{equation}
    \label{eq:rel_indicateur_pression_adim}
    \beta = \frac{1}{1 + \beta_0 \frac{{T^\star}^3}{\rho^\star}} = \frac{1}{1 + \beta_0 \frac{P_\mathrm{rad}^\star}{P_\mathrm{gaz}^\star}} = \frac{P_\mathrm{gaz}^\star}{P_\mathrm{gaz}^\star + \beta_0 P_\mathrm{rad}^\star}
\end{equation}

\subsection{Récapitulatif des adimensionnements}

Dans cette sous-partie, nous faisons un récapitulatif des variables et de leurs
adimensionnements, ainsi que des équations adimensionnées.

\subsubsection{Tableau récapitulatif des variables}

\begin{center}
    \begin{tabular}{|c|c|c|}
        \hline
        Variable & Variable adimensionnée & Adimensionnement \\
        \hline
        $\Omega$ & $\Omega^\star = \Omega/\Omega_0$ & $\Omega_0 = \left( \frac{G M}{r^3_s} \right)^\frac{1}{2}$ \\
        $t$ & $t^\star = t/t_0$ & $t_0 = 2 \Omega_0^{-1}$ \\
        $r$ & $x = \sqrt{r/r_s}$ & N/A \\
        $\nu$ & $\nu^\star = \nu/\nu_0$ & $\nu_0 = \frac{4 r_s^2}{3 t_0} = \frac{2 r_s^2 \Omega_0}{3}$ \\
        $v$ & $v^\star = v/v_0$ & $v_0 = 2 \frac{r_s}{t_0} = r_s \Omega_0$ \\
        $c_s$ & $c_s^\star = c_s/c_s^0$ & $c_s^0 = v_0$ \\
        $\Sigma$ & $\Sigma^\star = \Sigma/\Sigma_0$ & $\Sigma_0 = \frac{\dot{M_0}}{3 \pi \nu_0} = \frac{\dot{M_0}}{2 \pi r_s^2 \Omega_0}$ \\
        $S = \Sigma x$ & $S^\star = S/S_0 = \Sigma^\star x$ & $S_0 = \Sigma_0$ \\
        $H$ & $H^\star = H/H_0$ & $H_0 = r_s$ \\
        $\dot{M}$ & $\dot{M^\star} = \dot{M}/\dot{M_0}$ & N/A \\
        $\rho$ & $\rho^\star = \rho/\rho_0$ & $\rho_0 = \frac{\Sigma_0}{2 H_0} = \frac{\Sigma_0}{2 r_s}$ \\
        $T$ & $T^\star = T/T_0$ & $T_0 = \left(\frac{1}{\sqrt{27}} \times \frac{\dot{M_0} c^2}{48 \pi r_s^2 \sigma} \right)^\frac{1}{4}$ \\
        $F_z$ & $F_z^\star = F_z/F_z^0$ & $F_z^0 = \frac{\Sigma_0}{2 t_0} = \frac{\rho_0 H_0}{t_0} = \frac{\Sigma_0 \Omega_0}{4}$ \\
        $C_v$ & $C_v^\star = C_v/C_v^0$ & $C_v^0 = T_0^{-1}$ \\
        $P_\mathrm{gaz}$ & $P_\mathrm{gaz}^\star = P_\mathrm{gaz}/P_\mathrm{gaz}^0$ & $P_\mathrm{gaz}^0 = \frac{\rho_0 k_B T_0}{\mu m_p} = \rho_0 \frac{\mathcal{R} T_0}{\mu}$ \\
        $P_\mathrm{rad}$ & $P_\mathrm{rad}^\star = P_\mathrm{rad}/P_\mathrm{rad}^0$ & $P_\mathrm{rad}^0 = \frac{1}{3} a T_0^4$ \\
        \hline
    \end{tabular}
\end{center}

\subsubsection{Équations associées \label{sec::eq_adim}}

$x$ et $t$ sont les deux variables indépendantes. Pour les autres, certaines
sont des constantes du temps :
\begin{subequations}
    \begin{align}
        \mu &\approx 0.62 \\
        \Omega^\star &= x^{-3}
    \end{align}
\end{subequations}

Celles-ci (ainsi que $x$) ne seront calculées qu’une seule fois au début de la simulation.

Les autres non :
\begin{subequations}
    \begin{align}
        \frac{\partial S^\star}{\partial t^\star} &= \frac{1}{x^2} \frac{\partial^2}{\partial x^2} \left(\nu^\star S^\star\right) \label{eq:difS}\\
        v^\star &= - \frac{1}{S^\star x} \frac{\partial}{\partial x} \left(\nu^\star S^\star\right)\label{eq:difv} \\
        \nu^\star &= \alpha c_s^\star H^\star \\
        c_s^\star &= \Omega^\star H^\star \\
        \Sigma^\star &= \frac{S^\star}{x} \\
        \dot{M^\star} &= - x S^\star v^\star \label{eq:Mdotstar}\\
        \rho^\star &= \frac{S^\star}{x H^\star} \\
        C_v^\star \frac{\partial T^{\star}}{\partial t^{\star}} &=
        3 v_0^2 \nu^\star {\Omega^\star}^2 - \frac{F_z^\star x}{S^\star} +
        \frac{\mathcal{R} T_0}{\mu} \frac{4-3\beta}{\beta} \frac{T^\star}{S^\star} \times
        \left( \frac{\partial S^\star}{\partial t^\star} + v^\star \frac{\partial}{\partial x} \left(\frac{S^\star}{x}\right) \right) -
        \frac{C_v^\star v^\star}{x} \frac{\partial T^\star}{\partial x}\label{eq:difT} \\
        F_z^\star &=
        \begin{cases}
            \frac{2 c^2 x {T^\star}^4}{27 \sqrt{3} (\kappa_\mathrm{ff} + \kappa_e) S^\star \Sigma_0}, &\text{si $\tau_\mathrm{eff} \geq 1$} \\
            \frac{t_0}{\rho_0}\epsilon_\mathrm{ff} H^\star, &\text{si $\tau_\mathrm{eff} < 1$}
        \end{cases} \\
        \kappa_\mathrm{ff} &= \num{6.13e22} (\rho_0 \rho^\star) {T_0 T^\star}^{-\frac{7}{2}} \si{\square\centi\meter\per\gram} \\
        \kappa_e &= 0.2 \times (1 + X) \si{\square\centi\meter\per\gram} \approx \SI{0.34}{\square\centi\meter\per\gram} \\
        \epsilon_\mathrm{ff} &= \num{6.22e20} \times \rho^2 T^\frac{1}{2} \si{\cgs} \\
        \tau_\mathrm{eff} &= \frac{1}{2} (\kappa_e \kappa_\mathrm{ff})^\frac{1}{2} \frac{S^\star}{x} \Sigma_0 \\
        C_v^\star &= \frac{\mathcal{R} T_0}{\mu} \frac{12 (\gamma_g - 1)(1 - \beta) + \beta}{(\gamma_g - 1) \beta} \\
        P_\mathrm{gaz}^\star &= \rho^\star T^\star \\
        P_\mathrm{rad}^\star &= {T^\star}^4 \\
        \beta &= \frac{1}{1 + \beta_0 \frac{{T^\star}^3}{\rho^\star}} = \frac{P_\mathrm{gaz}^\star}{P_\mathrm{gaz}^\star + \beta_0 P_\mathrm{rad}^\star} ; \beta_0 = \frac{P_\mathrm{rad}^0}{P_\mathrm{gaz}^0}
    \end{align}
\end{subequations}

En pratique dans le code nous utiliserons $\Sigma$ là où c’est judicieux pour
éviter de calculer plusieurs fois $\nicefrac{S}{x}$.
