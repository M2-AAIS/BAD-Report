\section{intégration de $S^{\star}$ et $T^{\star}$}

\subsection{Méthode d'intégration}

Pour intégrer les équation de $S$ et $T$ (\eqref{eq:difS} et \eqref{eq:difT}), on choisit de réaliser une dérivée amont. Dans le cas présent, il s'agit d'une dérivée dans le sens croissant des indices du tableau. Ainsi pour calculer la dérivé partielle de $S^{\star}$ en $x$ à la case $i$ on calculera :

\begin{equation}
  \left. \frac{\partial S^{\star}}{\partial x} \right|_i = \frac{S^{\star}(i+1)-S^{\star}(i)}{\delta x} 
\end{equation}

Et pour calculer la dérivée seconde de $S^{\star}$ en $x$ à la case $i$ :

\begin{equation}
  \left. \frac{\partial^2 S^{\star}}{\partial x^2}\right|_i=\frac{S^{\star}(i+1)-2\ S^{\star}(i) +S^{\star}(i-1)}{\delta x^2} 
\end{equation}

Cependant, on constate qu'avec cette méthode il nous faut connaître la valeur de la case $i+1$, case inexistante dans le cas $i=i_{max}$.

\subsection{Condition au bords de $\nu^{\star}S^{\star}$}

\subsubsection{Dérivée partielle première}

Dans l'équation différentielle sur $v^{\star}$ \eqref{eq:difv} on doit connaître $\frac{\partial \nu^{\star} S^{\star}}{\partial x}$ en $i=i_{max}$.
On fait l'hypothèse qu'en $r_{max}$ le taux d'accrétion est $\dot{M}_0$. Dés lors par \eqref{eq:Mdotstar} on a $(\dot{M}^{\star}_0=)1 = -(x S^{\star}v^{\star})$ et par \eqref{eq:difv} il vient :

\begin{equation}
  \label{eq:nuS_n_is_null}
  \left. \frac{\partial (\nu^{\star} S^{\star})}{\partial x}\right|_{i=i_{max}}=1
\end{equation}


\subsubsection{dérivée partielle seconde}
Afin de calculer $\frac{\partial^2(\nu^{\star} S^{\star})} {\partial x^2}$ en $i=i_{max}$ il nous faut connaître les valeurs hypothétiques de $(\nu^{\star} S^{\star})(i=i_{max}+1)$ que l'on va obtenir à partir des valeurs des dérivées premières. On a donc :

\begin{eqnarray}
  \left. \frac{\partial (\nu^{\star}S^{\star})}{\partial x} \right|_{i_{max}} = 1 = \frac{(\nu^{\star} S^{\star})(i_{max}+1)-(\nu^{\star}S^{\star})(i_{max})}{\delta x} 
\end{eqnarray}

Dès lors on a : 
\begin{equation}
( \nu^{\star} S^{\star})(i_{max}+1) = \delta x + (\nu^{\star}S^{\star})(i_{max}) 
\end{equation}

Il vient donc :
\begin{eqnarray}
 \left. \frac{\partial^2 (\nu^{\star} S^{\star})}{\partial x^2}\right|_{i=i_{max}} &=& \frac{(\nu^{\star} S^{\star})(i_{max}+1) - 2\ (\nu^{\star} S^{\star})(i_{max}) + (\nu^{\star} S^{\star})(i_{max}-1)}{\delta x^2}\\
 \left. \frac{\partial^2 (\nu^{\star} S^{\star})}{\partial x^2}\right|_{i=i_{max}} &=& \frac{\delta x - (\nu^{\star} S^{\star})(i_{max}) + (\nu^{\star} S^{\star})(i_{max}-1)}{\delta x^2}
\end{eqnarray}

\subsection{Conditions aux bords de $S^{\star}$}
On cherche les conditions aux bords de $\frac{\partial S^{\star}}{\partial t^{\star}}$, pour cela on utilise l'équation \eqref{eq:difS}. Dès lors on a directement :

\begin{eqnarray}
  \left. \frac{\partial S^{\star}}{\partial t^{\star}} \right|_{i=i_{max}}&= \frac{1}{(x(i_{max}))^2} \left. \frac{\partial^2 (\nu^{\star} S^{\star})}{\partial x^2}\right|_{i=i_{max}} &= \frac{1}{(x(i_{max}))^2} \frac{\delta x - (\nu^{\star} S^{\star})(i_{max}) + (\nu^{\star} S^{\star})(i_{max}-1)}{\delta x^2}
\end{eqnarray}


\subsection{Condition aux bords de $S^{\star}/x$}

Dans l'équation \eqref{eq:difT} il est nécessaire de connaître les valeurs $\frac{\partial}{\partial x}\left(\frac{S^{\star}}{x}\right)$ en chacun des points du disque, donc en $i=i_{max}$. 

%FIXME
On a : (je ne sais plus l'argument !!!)
\begin{equation}
  \frac{\partial S^{\star}}{\partial t^{\star}} + v^{\star} \frac{\partial}{\partial x} \left(\frac{S^{\star}}{x}\right)=0
\end{equation}

Donc avec \eqref{eq:difT}

\begin{equation}
  \frac{\partial}{\partial x} \left(\frac{S^{\star}}{x}\right) = -\frac{1}{x^2\ v^{\star}} \frac{\partial^2 (\nu^{\star} S^{\star})}{\partial x^2}
\end{equation}

dés lors on a :

\begin{equation}
\left. \frac{\partial}{\partial x} \left(\frac{S^{\star}}{x}\right) \right|_{i_{max}} = \left. -\frac{1}{x^2\ v^{\star}} \frac{\partial^2 (\nu^{\star} S^{\star})}{\partial x^2} \right|_{i_{max}} = \frac{1}{(x^2\ v^{\star})(i_{max})} \frac{\delta x - (\nu^{\star} S^{\star})(i_{max}) + (\nu^{\star} S^{\star})(i_{max}-1)}{\delta x^2}
\end{equation}


