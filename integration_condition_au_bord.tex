\section{integration de $S^{\star}$ et $T^{\star}$ et condition au bord}

\subsection{Méthode d'intégration}

Pour intégrer les équation de S et T (\eqref{eq:difS} et \eqref{eq:difT}), étant donné que ce ne sont pas des équation de transport, nous avons le choix du sens d'intégration. Afin de gagner en précision, pour la dérivée en un point on fera la moyenne des dérivé a droite et a gauche. Ainsi pour calculer la dérivé partielle de $S^{\star}$ en x a la case $i$ on calculera :

\begin{equation}
\left. \frac{\partial S^{\star}}{\partial x} \right|_i = \frac{S^{\star}(i+1)-S^{\star}(i-1)}{2\delta x} \label{eq:dif_ordre1}
\end{equation}

Et pour calculer la dérivée seconde de $S^{\star}$ par en x a la case $i$ :

\begin{equation}
\left. \frac{\partial^2 S^{\star}}{\partial x^2}\right|_i=\frac{S^{\star}(i+1)-2\ S^{\star}(i) +S^{\star}(i-1)}{\delta x^2} \label{eq:dif_ordre2}
\end{equation}

Cependant, on constate qu'avec cette méthode il est nécessaire de connaitre la valeur des cases voisines, or la première ($i=1$) et la dernière ($i=i_{max}$) n'ont pas de voisin de part et d'autre, il nous faut donc déterminer des condition au bord.


\subsection{Condition au bords de $\nu^{\star}S^{\star}$}

\subsubsection{dérivée partielle première}

Dans l'équation différentielle sur $v^{\star}$ \eqref{eq:difv} on doit connaitre $\frac{\partial \nu^{\star} S^{\star}}{\partial x}$ en $i=1$ et $i=i_{max}$.

\paragraph{$i=1$}
Dans ce projet on fais l'hypothèse que en $r_{min}$ toute la matière est accréter par le trou noir donc $\Sigma(r_{min})=0$ de pars l'adimensionnement que l'on a pris il vient $S^{\star}(i=1)=0$ dés lors par l'équation \eqref{eq:difv} on a

\begin{equation}
\left. \frac{\partial (\nu^{\star} S^{\star})}{\partial x}\right|_{i=1}=0\label{eq:dif_nuS_i1}
\end{equation}

\paragraph{$i=i_{max}$}
On fais l'hypothèse que en $r_{max}$ le taux d'accrétion est $\dot{M}_0$ dés lors par \eqref{eq:Mdotstar} on a $(\dot{M}^{\star}_0=)1 = -(x S^{\star}v^{\star})$ et par \eqref{eq:difv} il vient :

\begin{equation}
\left. \frac{\partial (\nu^{\star} S^{\star})}{\partial x}\right|_{i=i_{max}}=1\label{eq:dif_nuS_imax}
\end{equation}

\subsubsection{dérivée partielle seconde}
Afin de calculé $\frac{\partial^2(\nu^{\star} S^{\star})} {\partial x^2}$ en $i=1$ et $i=i_{max}$ il nous faut connaitre les valeur hypothétique de $(\nu^{\star} S^{\star})(i=1-1)$ et $(\nu^{\star} S^{\star})(i=i_{max}+1)$ que l'on va obtenir a partir des valeurs des dérivée première.

\paragraph{i=1}
On a d'après \eqref{eq:dif_ordre1} et \eqref{eq:dif_nuS_i1} :
\begin{equation}
\left. \frac{\partial (\nu^{\star} S^{\star})}{\partial x}\right|_{i=1}=0=\frac{ (\nu^{\star} S^{\star})(1+1) - (\nu^{\star} S^{\star})(1-1)}{2\delta x}
\end{equation}

dès lors
\begin{equation}
(\nu^{\star} S^{\star})(1-1)=(\nu^{\star} S^{\star})(1+1)
\end{equation}

On en déduit donc la condition au bord :

\begin{eqnarray}
\left. \frac{\partial^2 (\nu^{\star} S^{\star})}{\partial x^2}\right|_{i=1} &=& \frac{(\nu^{\star} S^{\star})(1+1) -2\ \overbrace{(\nu^{\star} S^{\star})(1)}^{=0}+ (\nu^{\star} S^{\star})(1-1)}{\delta x ^2} \\
\left. \frac{\partial^2 (\nu^{\star} S^{\star})}{\partial x^2}\right|_{i=1} &=& \frac{2\ (\nu^{\star} S^{\star})(2)}{\delta x ^2}
\end{eqnarray}

\paragraph{$i=i_{max}$}
De même que pour $i=1$ on utilise la valeur de la dérivée première pour calculer $(\nu^{\star} S^{\star})(i_{max}+1)$, on obtient :
\begin{equation}
(\nu^{\star} S^{\star})(i_{max}+1)=2\delta x + (\nu^{\star} S^{\star})(i_{max}-1)
\end{equation}

Il vient donc 
\begin{eqnarray}
\left. \frac{\partial^2 (\nu^{\star} S^{\star})}{\partial x^2}\right|_{i=i_{max}} &=& \frac{(\nu^{\star} S^{\star})(i_{max}+1) - 2\ (\nu^{\star} S^{\star})(i_{max}) + (\nu^{\star} S^{\star})(i_{max}-1)}{\delta x^2}\\
\left. \frac{\partial^2 (\nu^{\star} S^{\star})}{\partial x^2}\right|_{i=i_{max}} &=& 2\ \frac{\delta x + (\nu^{\star} S^{\star})(i_{max}-1)- (\nu^{\star} S^{\star})(i_{max})}{\delta x^2}
\end{eqnarray}

\subsection{conditions aux bords de $S^{\star}$}
On cherche les condition au bords de $\frac{\partial S^{\star}}{\partial t^{\star}}$, pour cela on utilise l'équation \eqref{eq:difS}. Dès lors on a directement :

\begin{eqnarray}
\left. \frac{\partial S^{\star}}{\partial t^{\star}} \right|_{i=1} &= \frac{1}{(x(1))^2} \left. \frac{\partial^2 (\nu^{\star} S^{\star})}{\partial x^2}\right|_{i=1} &= \frac{1}{(x(1))^2} \frac{2\ (\nu^{\star} S^{\star})(2)}{\delta x ^2}\\
\left. \frac{\partial S^{\star}}{\partial t^{\star}} \right|_{i=i_{max}}&= \frac{1}{(x(i_{max}))^2} \left. \frac{\partial^2 (\nu^{\star} S^{\star})}{\partial x^2}\right|_{i=i_{max}} &= \frac{2}{(x(i_{max}))^2} \frac{\delta x + (\nu^{\star} S^{\star})(i_{max}-1)- (\nu^{\star} S^{\star})(i_{max})}{\delta x^2}
\end{eqnarray}

\subsection{condition aux bords de $S^{\star}/x$}

Dans l'équation \eqref{eq:difT} il est nécessaire de connaitre les valeur $\frac{\partial}{\partial x}\left(\frac{S^{\star}}{x}\right)$ en chacun des point du disque, donc a chacune des extrémités. 

%FIXME
On a : (je ne sais plus l'argument !!!)
\begin{equation}
\frac{\partial S^{\star}}{\partial t^{\star}} + v^{\star} \frac{\partial}{\partial x} \left(\frac{S^{\star}}{x}\right)=0
\end{equation}

Donc avec \eqref{eq:difT}

\begin{equation}
\frac{\partial}{\partial x} \left(\frac{S^{\star}}{x}\right) = -\frac{1}{x^2\ v^{\star}} \frac{\partial^2 (\nu^{\star} S^{\star})}{\partial x^2}
\end{equation}

On en déduit donc :

\begin{eqnarray}
\left. \frac{\partial}{\partial x} \left(\frac{S^{\star}}{x}\right) \right|_{i=1 }&=& \frac{1}{(x^2\ v^{\star})(1)} \frac{2\ (\nu^{\star} S^{\star})(2)}{\delta x ^2} \\
\left. \frac{\partial}{\partial x} \left(\frac{S^{\star}}{x}\right) \right|_{i=i_{max}} &=& \frac{2}{(x^2\ v^{\star})(i_{max})} \frac{\delta x + (\nu^{\star} S^{\star})(i_{max}-1)- (\nu^{\star} S^{\star})(i_{max})}{\delta x^2}
\end{eqnarray}