\section{Intégration de $S^{\star}$ et $T^{\star}$ (mauvaise façon) et conditions aux bords}

\subsection{Méthode d'intégration}

Pour intégrer les équations de S et T (\eqref{eq:difS} et \eqref{eq:difT}), étant donné que ce ne sont pas des équations de transport, nous avons le choix du sens d'intégration. Afin de gagner en précision, pour la dérivée en un point on fera la moyenne des dérivées à droite et à gauche. Ainsi pour calculer la dérivée partielle de $S^{\star}$ en x à la case $i$ on calculera :

\begin{equation}
  \left. \frac{\partial S^{\star}}{\partial x} \right|_i = \frac{S^{\star}(i+1)-S^{\star}(i-1)}{2\delta x} \label{eq:dif_ordre1}
\end{equation}

Et pour calculer la dérivée seconde de $S^{\star}$ par en x à la case $i$ :

\begin{equation}
  \left. \frac{\partial^2 S^{\star}}{\partial x^2}\right|_i=\frac{S^{\star}(i+1)-2\ S^{\star}(i) +S^{\star}(i-1)}{\delta x^2} \label{eq:dif_ordre2}
\end{equation}

Cependant, on constate qu'avec cette méthode il est nécessaire de connaître la valeur des cases voisines, or la première ($i=1$) et la dernière ($i=i_{max}$) n'ont pas de voisins de part et d'autre, il nous faut donc déterminer des conditions aux bords.


\subsection{Conditions aux bords de $\nu^{\star}S^{\star}$}

\subsubsection{Dérivée partielle première}

Dans l'équation différentielle sur $v^{\star}$ \eqref{eq:difv} on doit connaître $\frac{\partial \nu^{\star} S^{\star}}{\partial x}$ en $i=1$ et $i=i_{max}$.

\paragraph{$i=1$}
Dans ce projet on fait l'hypothèse que juste après $r_{min}$ toute la matière est accrétée par le trou noir donc $S^{\star}(i=1-1)=0$ dés lors par l'équation \eqref{eq:difv} on a

\begin{equation}
  \left. \frac{\partial (\nu^{\star} S^{\star})}{\partial x}\right|_{i=1}=\frac{(\nu^\star S^{\star})(1+1)}{2 \delta x}\label{eq:dif_nuS_i1}
\end{equation}

\paragraph{$i=i_{max}$}
On fait l'hypothèse qu' en $r_{max}$ le taux d'accrétion est $\dot{M}_0$ dès lors par \eqref{eq:Mdotstar} on a $(\dot{M}^{\star}_0=)1 = -(x S^{\star}v^{\star})$ et par \eqref{eq:difv} il vient :

\begin{equation}
  \left. \frac{\partial (\nu^{\star} S^{\star})}{\partial x}\right|_{i=i_{max}}=1\label{eq:dif_nuS_imax}
\end{equation}

\subsubsection{Dérivée partielle seconde}
Afin de calculer $\frac{\partial^2(\nu^{\star} S^{\star})} {\partial x^2}$ en $i=1$ et $i=i_{max}$ il nous faut connaître les valeurs hypothétiques de $(\nu^{\star} S^{\star})(i=1-1)$ et $(\nu^{\star} S^{\star})(i=i_{max}+1)$ que l'on va obtenir à partir des valeurs des dérivées premières.

\paragraph{i=1}
On a d'après \eqref{eq:dif_ordre1} et \eqref{eq:dif_nuS_i1} :
\begin{equation}
  \left. \frac{\partial (\nu^{\star} S^{\star})}{\partial x}\right|_{i=1}=0=\frac{ (\nu^{\star} S^{\star})(1+1) - (\nu^{\star} S^{\star})(1-1)}{2\delta x}
\end{equation}

dès lors
\begin{equation}
  (\nu^{\star} S^{\star})(1-1)=(\nu^{\star} S^{\star})(1+1)
\end{equation}

On en déduit donc la condition au bord :

\begin{eqnarray}
  \left. \frac{\partial^2 (\nu^{\star} S^{\star})}{\partial x^2}\right|_{i=1} &=& \frac{(\nu^{\star} S^{\star})(1+1) -2\ \overbrace{(\nu^{\star} S^{\star})(1)}^{=0}+ (\nu^{\star} S^{\star})(1-1)}{\delta x ^2} \\
  \left. \frac{\partial^2 (\nu^{\star} S^{\star})}{\partial x^2}\right|_{i=1} &=& \frac{2\ (\nu^{\star} S^{\star})(2)}{\delta x ^2}
\end{eqnarray}

\paragraph{$i=i_{max}$}
De même que pour $i=1$ on utilise la valeur de la dérivée première pour calculer $(\nu^{\star} S^{\star})(i_{max}+1)$, on obtient :
\begin{equation}
  (\nu^{\star} S^{\star})(i_{max}+1)=2\delta x + (\nu^{\star} S^{\star})(i_{max}-1)
\end{equation}

Il vient donc 
\begin{eqnarray}
  \left. \frac{\partial^2 (\nu^{\star} S^{\star})}{\partial x^2}\right|_{i=i_{max}} &=& \frac{(\nu^{\star} S^{\star})(i_{max}+1) - 2\ (\nu^{\star} S^{\star})(i_{max}) + (\nu^{\star} S^{\star})(i_{max}-1)}{\delta x^2}\\
  \left. \frac{\partial^2 (\nu^{\star} S^{\star})}{\partial x^2}\right|_{i=i_{max}} &=& 2\ \frac{\delta x + (\nu^{\star} S^{\star})(i_{max}-1)- (\nu^{\star} S^{\star})(i_{max})}{\delta x^2}
\end{eqnarray}

\subsection{Conditions aux bords de $S^{\star}$}
On cherche les conditions aux bords de $\frac{\partial S^{\star}}{\partial t^{\star}}$, pour celà on utilise l'équation \eqref{eq:difS}. Dès lors on a directement :

\begin{eqnarray}
  \left. \frac{\partial S^{\star}}{\partial t^{\star}} \right|_{i=1} &= \frac{1}{(x(1))^2} \left. \frac{\partial^2 (\nu^{\star} S^{\star})}{\partial x^2}\right|_{i=1} &= \frac{1}{(x(1))^2} \frac{2\ (\nu^{\star} S^{\star})(2)}{\delta x ^2}\\
  \left. \frac{\partial S^{\star}}{\partial t^{\star}} \right|_{i=i_{max}}&= \frac{1}{(x(i_{max}))^2} \left. \frac{\partial^2 (\nu^{\star} S^{\star})}{\partial x^2}\right|_{i=i_{max}} &= \frac{2}{(x(i_{max}))^2} \frac{\delta x + (\nu^{\star} S^{\star})(i_{max}-1)- (\nu^{\star} S^{\star})(i_{max})}{\delta x^2}
\end{eqnarray}

\subsection{Conditions aux bords de $S^{\star}/x$}

Dans l'équation \eqref{eq:difT} il est nécessaire de connaître les valeurs $\frac{\partial}{\partial x}\left(\frac{S^{\star}}{x}\right)$ en chacun des points du disque, donc à chacune des extrémités. 

%FIXME
On a : (je ne sais plus l'argument !!!)
\begin{equation}
  \frac{\partial S^{\star}}{\partial t^{\star}} + v^{\star} \frac{\partial}{\partial x} \left(\frac{S^{\star}}{x}\right)=0
\end{equation}

Donc avec \eqref{eq:difT}

\begin{equation}
  \frac{\partial}{\partial x} \left(\frac{S^{\star}}{x}\right) = -\frac{1}{x^2\ v^{\star}} \frac{\partial^2 (\nu^{\star} S^{\star})}{\partial x^2}
\end{equation}

On en déduit donc :

\begin{eqnarray}
  \left. \frac{\partial}{\partial x} \left(\frac{S^{\star}}{x}\right) \right|_{i=1 }&=& - \frac{1}{(x^2\ v^{\star})(1)} \frac{2\ (\nu^{\star} S^{\star})(2)}{\delta x ^2} \\
  \left. \frac{\partial}{\partial x} \left(\frac{S^{\star}}{x}\right) \right|_{i=i_{max}} &=& - \frac{2}{(x^2\ v^{\star})(i_{max})} \frac{\delta x + (\nu^{\star} S^{\star})(i_{max}-1)- (\nu^{\star} S^{\star})(i_{max})}{\delta x^2}
\end{eqnarray}
